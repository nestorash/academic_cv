\documentclass[11pt]{article}
\usepackage[document]{ragged2e}
\usepackage{url,fancyhdr,enumerate,setspace}
\usepackage[letterpaper]{geometry}
\usepackage[utf8]{inputenc}
\usepackage[greek]{babel}
\usepackage[LGR,T1]{fontenc}
\usepackage{revnum}
\usepackage{kerkis}
\geometry{verbose,tmargin=1in,bmargin=1in,lmargin=1in,rmargin=1in}


\newcommand{\en}{\textlatin} %Convenient language switch
%\usepackage[breaklinks,pdfnewwindow]{hyperref} %Compile with PDF LaTeX ***


\begin{document}

\setcounter{page}{0}
\thispagestyle{empty}
\makeatletter
\def\maketitle{%
  \null
  \thispagestyle{empty}%
  \vfill
  \begin{center}\leavevmode
    \normalfont
    \vskip 4cm
    {\huge \@title\par}%
    \vskip 1.4cm
    {\Large \@author\par}
    \vskip 1.02cm
    {\Large \@location\par}%
    \vskip 0.02cm
    {\Large \@CHA\par}%
    \vskip 6cm
    {\LARGE \@date\par}%
  \end{center}%
  \vfill
  \null
  \cleardoublepage
  }
  \def\location#1{\def\@location{#1}}
 \def\addressGOT#1{\def\@addressGOT{#1}}
 \def\CHA#1{\def\@CHA{#1}}
\makeatother

\centering \author{\Huge{\textbf{Νέστωρ Χατζηδιαμαντής}
\newline\Large {Διπλ. Μηχ., \latintext PhD}}}
\title{\textbf{\LARGE{\textbf{ΒΙΟΓΡΑΦΙΚΟ ΣΗΜΕΙΩΜΑ ΚΑΙ ΥΠΟΜΝΗΜΑ ΔΗΜΟΣΙΕΥΜΕΝΩΝ ΕΡΓΑΣΙΩΝ}}}}

\location{}
\addressGOT{}
\CHA{}
\date{\today}
%\end{spacing}
\maketitle
\tableofcontents
\setcounter{tocdepth}{3}
\newpage
\noindent \begin{center}
 \textbf{\LARGE Νέστωρ Χατζηδιαμαντής}\textbf{ \vspace{2mm}
}\\
% \latintext \textsc{PhD, Dipl.-Ing.}  \textbf{}\\
 \latintext {e-mail: %\href{mailto:nestorash@gmail.com}
 {nestorash@gmail.com}}\textbf{}\\
\textbf{}\\
\textbf{ }
\par\end{center}
\noindent
\parbox[t][0pt]{8cm}{%
\textbf{\large Προσωπικά στοιχεία}{\small }\\
{\small Ημερομηνία Γέννησης: 22 Ιανουαρίου 1981}\\
{\small Τόπος γέννησης: Γκλέντεϊλ, Η.Π.Α.}\\
{\small Στρατιωτικές Υποχρεώσεις: Εκπληρωμένες}\\
{\small Οικογενειακή κατάσταση: Έγγαμος με δύο παιδιά}{\vspace*{1mm}
}
} \hfill{}%
\parbox[t][0pt]{7.5cm}{%
\textbf{\large Διεύθυνση Κατοικίας}{\small }\\
{\small}Κανάρη 16,
{\small Πεύκη, 15121, Αττική}\\
{\small Τηλ: +30 6942712099}\\
}
%\vskip 0.1cm
\noindent \rule[-1.1in]{6.5in}{1pt}

\pagestyle{fancy}
\rhead{\greektext Βιογραφικό Σημείωμα}
\lhead{\greektext Νέστωρ Χατζηδιαμαντής}
\justify
\section{ΓΕΝΙΚΕΣ ΠΛΗΡΟΦΟΡΙΕΣ}

\subsection{Σπουδές - Ακαδημαϊκοί Τίτλοι}
\begin{itemize}
\item Διδακτορικό Δίπλωμα, Τμήμα Ηλεκτρολόγων Μηχανικών και Μηχανικών Η/Υ, \\Αριστοτέλειο Πανεπιστήμιο Θεσσαλονίκης\hfill{}09/2008-04/2012
\item Μεταπτυχιακό δίπλωμα \latintext{MSc} \greektext{σε Τηλεπικοινωνιακά Δίκτυα και Λογισμικό,} \\ \latintext{University of Surrey} \hfill{}09/2005-09/2006
\item  \greektext{Δίπλωμα Ηλεκτρολόγου Μηχανικού και Μηχανικού Η/Υ,  Τμήμα Ηλεκτρολόγων Μηχανικών και Μηχανικών Η/Υ, \\Αριστοτέλειο Πανεπιστήμιο Θεσσαλονίκης}
\hfill{}09/1999-03/2005
\end{itemize}

\subsection{Επαγγελματική και Ερευνητική Απασχόληση}
\begin{itemize}

\item \textbf{ΔΕΔΔΗΕ Α.Ε.}\hfill{}01/2016-Παρόν\\
\emph{Θέση:} Εντεταλμένος Μηχανικός, Τομέας Κατασκευών Περιοχής Αθήνας\\
\emph{Περιγραφή:} Εκπόνηση μελετών δικτύων διανομής ενέργειας, επίβλεψη Αναδόχου, προμήθεια υλικών
%\emph{Group}: Head of Wireless Communications Systems Group (WCSG)

\item \textbf{Αριστοτέλειο Πανεπιστήμιο Θεσσαλονίκης}\hfill{}06/2012-10/2015\\
\emph{Θέση:} Μεταδιδακτορικός Ερευνητής\\
\emph{Τμήμα:} Ηλεκτρολόγων Μηχανικών και Μηχανικών Η/Υ

\item \textbf{Ινστιτούτο Πληροφορικής και Τηλεματικής, ΕΚΕΤΑ}\hfill{}01/2013-12/2013\\
\emph{Θέση}: Μεταδιδακτορικός Ερευνητής
\end{itemize}
\section{Ερευνητική Δραστηριότητα και Διακρίσεις}

\subsection{Ερευνητικά Ενδιαφέροντα}
\justify{Τα τρέχοντα ερευνητικά ενδιαφέροντα περιλαμβάνουν:}
\begin{itemize}
\item Ασύρματες Οπτικές Επικοινωνίες (\latintext{Optical Wireless Communications})
\item \greektext Θεωρία τηλεπικοινωνιακών συστημάτων (\latintext{Communication Theory})
\item \greektextΣυστήματα Επικοινωνιών Πολλαπλών Εισόδων και Εξόδων (\latintext{Multiple Input Multiple Output Communications})
\item \greektext Συνεργατικές επικοινωνίες (\latintext{Cooperative Communications})
\item \greektext Γνωστικές ραδιοεπικοινωνίες (\latintext{Cognitive Radio})
\end{itemize}

\subsection{Συμμετοχή σε Ερευνητικά Προγράμματα}
\begin{itemize}

\item  \greektext «Εξόρυξη και εκμετάλλευση κοινωνικής δομής υποδικτύων για αποδοτική λειτουργία γνωστικών συστημάτων» (\latintext SOCONET), \greektext Πρόγραμμα ΘΑΛΗΣ \\ \textbf{Φορέας Διαχείρισης:} ΕΛΚΕ ΑΠΘ, Επιστ. Υπεύθυνος Λ. Γεωργιάδης.

\item \latintext "Cooperative Networking for High Capacity Transport Architectures'' (CONECT) \greektext \\ \textbf{Φορέας Διαχείρισης:} ΕΚΕΤΑ, Επιστ. Υπεύθυνος: Λ. Τασιούλας.

\item \latintext "Statistical Mechanics Inspired Methods for Green Autonomous Networking'' (STAMINA) \greektext \\ \textbf{Φορέας Διαχείρισης:} ΕΚΕΤΑ, Επιστ. Υπεύθυνος: Λ. Τασιούλας. \end{itemize}


\subsection{Διακρίσεις}
\begin{itemize}

\item \greektext Υποτροφία Αριστείας για Μεταδιδάκτορες από την Επιτροπή Ερευνών Α.Π.Θ.

\item Παραδειγματικός κριτής του επιστημονικού περιοδικού \latintext{IEEE Communications Letters}.

\item \greektext Υποτροφία Αριστείας Υποψήφιων Διδακτόρων από την Επιτροπή Ερευνών Α.Π.Θ.

\item Βραβείο της \latintext{ERICSSON} \greektext για καλύτερη διπλωματική εργασία στον χώρο των τηλεπικοινωνιών (\latintext{''ERICSSON's excellence awards"}).

\end{itemize}

\subsection{\greektext{Αναφορές στο Δημοσιευμένο Έργο την 7/2/2017}}
\begin{itemize}\greektext

\item \latintext GOOGLE SCHOLAR:
\begin{itemize}
    \item 601 \greektext{με} $h=14$ (\greektext περιλαμβάνει αυτοαναφορές).
\end{itemize}

\item \latintext SCOPUS:
\begin{itemize}
\item  414 \greektext{με} $h=11$
\end{itemize}
\end{itemize}
\section{\greektext{Διδακτική Εμπειρία}}

\subsection{\greektext{Διδασκαλία Προπτυχιακών Μαθημάτων στο ΤΗΜΜΥ-ΑΠΘ ως Επικουρών}}
\begin{itemize}\greektext

\item Χειμερινό εξάμηνο 2011-2012: Κινητές και Δορυφορικές Επικοινωνίες.

\item Θερινό εξάμηνο 2010-2011: Ψηφιακές Τηλεπικοινωνίες I.

\item Χειμερινό εξάμηνο 2010-2011: Κινητές και Δορυφορικές Επικοινωνίες, Αναλογικές Τηλεπικοινωνίες.

\item Θερινό εξάμηνο 2009-2010: Ψηφιακές Τηλεπικοινωνίες I.

\item Χειμερινό εξάμηνο 2009-2010: Δορυφορικές Επικοινωνίες, Αναλογικές Τηλεπικοινωνίες.

\item Θερινό εξάμηνο 2008-2009: Ψηφιακές Τηλεπικοινωνίες I. \end{itemize}

\subsection{\greektext{Αυτοδύναμη Διδασκαλία}}
\begin{itemize}
\item \greektext Θερινό εξάμηνο 2014-2015: Δίκτυα Κινητών Επικοινωνιών.\\ Τμήμα Μηχανικών Πληροφορικής και Τηλεπικοινωνιών, Πανεπιστήμιο Δυτικής Μακεδονίας
\item \greektext Θερινό εξάμηνο 2012-2013: Δίκτυα Κινητών Επικοινωνιών. \\ Τμήμα Μηχανικών Πληροφορικής και Τηλεπικοινωνιών, Πανεπιστήμιο Δυτικής Μακεδονίας
    \end{itemize}

\justify \textbf{Ύλη Μαθήματος:} Βασικές Αρχές Δικτύων Κινητών Επικοινωνιών. Διάδοση και Παρεμβολές. Αρχιτεκτονική Κυψελωτών Συστημάτων. Συστήματα Κινητών Επικοινωνιών 2ης, 2.5ης και 3ης Γενιάς. Συστήματα 4ης Γενιάς. Βασικές Λειτουργίες Δικτύων Κινητών Επικοινωνιών. Αρχές Σχεδίασης Δικτύων Κινητών Επικοινωνιών. Τεχνικές Ανάθεσης Πόρων. Διαχείριση Ραδιοδιαύλων. Διαχείριση Κινητικότητας. Αλγοριθμικές Τεχνικές Διαπομπής. Συστήματα Σηματοδοσίας.
\section{\greektext{Επαγγελματική και Επιστημονική Δραστηριότητα}}

\subsection{\greektext{Συμμετοχή σε Επαγγελματικές και Επιστημονικές Οργανώσεις}}

\begin{itemize}

\item \latintext Member of IEEE Communications Society

\item Member of OSA

\item \greektext Μέλος Τεχνικού Επιμελητηρίου Ελλάδος
\end{itemize}

\greektext{}

\greektext
\subsection{Κριτής σε Διεθνή Επιστημονικά Περιοδικά}
Από το 2008 έχει κρίνει περισσότερα από 70 άρθρα για τα εξής επιστημονικά περιοδικά: \latintext{IEEE Transactions on
Communications, IEEE Journal on Selected Areas in Communications, IEEE Transactions on Wireless Communications, IEEE Transactions on Vehicular Technology, IEEE Transactions on Signal Processing,
OSA/IEEE Journal of  Lightwave Technology, IEEE Signal Processing Letters, IEEE Antennas and Propagation Letters, IEEE
Communications Letters, IET Communications, IET Electronics Letters, IET Signal Processing, EURASIP Journal on Wireless
Communications and Networking, Journal of Communications and Networks}.

\greektext
\section{\greektext Δημοσιευμένο Έργο}

\subsection{\greektext Διδακτορική Διατριβή}
\textbf{Τίτλος}: «Ανάπτυξη και Αξιολόγηση της Επίδοσης Ασύρματων Συστημάτων Επικοινωνίας Υψηλής Χωρητικότητας», Αριστοτέλειο Πανεπιστήμιο Θεσσαλονίκης, Μάρτιος 2012.\\
\textbf{Επιβλέπων}: Καθηγητής Γ. Κ. Καραγιαννίδης

\greektext
\subsection{Διεθνή Επιστημονικά Περιοδικά με Κριτές}
\renewcommand{\labelenumi}{[J\arabic{enumi}]}
\begin{enumerate}\latintext
\item H. G. Sandalidis, \textbf{N. D. Chatzidiamantis}, G. D. Ntouni, and G. K. Karagiannidis, "Performance of an FSO Link using a New Mixture Composite Irradiance Model", \emph{IET Electronics Letters} To appear

\item H. G. Sandalidis, \textbf{N. D. Chatzidiamantis},
and G. K. Karagiannidis, \textquotedblleft A Tractable Model for Turbulence and Misalignment Induced Fading in Optical Wireless Systems,\textquotedblright\ \emph{IEEE Communications Letters}, vol. 20, no. 9, pp. 1904--1907, 2016.
\item	A. -A. A. Boulogeorgos, \textbf{N. D. Chatzidiamantis}, and G. K. Karagiannidis, \textquotedblleft Energy Detection Spectrum Sensing Under RF imperfections,\textquotedblright\ \emph{IEEE Transactions on Communications}, vol. 64, no. 7, pp. 2754--2766, 2016.
\item A. -A. A. Boulogeorgos, \textbf{N. D. Chatzidiamantis}, and G. K. Karagiannidis, \textquotedblleft Spectrum Sensing with Multiple Primary Users over Fading Channels,\textquotedblright\ \emph{IEEE Communications Letters}, vol. 20, no. 7, pp. 1457--1460, 2016.
\item	P. Puri, \textbf{N. D. Chatzidiamantis}, P. Garg, M. Aggarwal, and G. K. Karagiannidis, \textquotedblleft Two-Way Relay Selection in Multiple Relayed FSO Networks,\textquotedblright\ \emph{IEEE Wireless Communications Letters}, vol. 4, no. 5, pp. 485--488, 2015.

\item	\textbf{N. D. Chatzidiamantis}, L. Georgiadis, H. Sandalidis and G. K. Karagiannidis, \textquotedblleft Throughput-Optimal Link-Layer Design in Power Constrained Hybrid OW/RF Systems,\textquotedblright\ \emph{IEEE Journal on Selected Areas in Communications}, vol. 33, no. 9, pp. 1972--1984, 2015.
\item	\textbf{N. D. Chatzidiamantis}, E. Matskani, L. Georgiadis, I. Koutsopoulos, L. Tassiulas, \textquotedblleft Optimal Primary-Secondary User Cooperation Policies in Cognitive Radio Networks,\textquotedblright \emph{IEEE Transactions on Wireless Communications}, vol. 14, No 6, pp. 3443-3455, Feb. 2015.
\item K. N. Pappi, \textbf{N. D. Chatzidiamantis}, and G. K. Karagiannidis, \textquotedblleft Error
Performance of Multidimensional Lattice Constellations-Part II: Evaluation over Fading
Channels,\textquotedblright\ \emph{IEEE Transactions on Communications}, vol. 61, no. 3, pp. 1099--1110, 2013.

\item K. N. Pappi, \textbf{N. D. Chatzidiamantis}, and G. K. Karagiannidis, Error
Performance of Multidimensional Lattice Constellations-Part I: A Parallelotope Geometry Based Approach for the AWGN
Channel,\textquotedblright\ \emph{IEEE Transactions on Communications}, vol. 61, no. 3, pp. 1088--1098, 2013.

\item D. S. Michalopoulos, \textbf{N. D. Chatzidiamantis}, R. Schober, and G. K. Karagiannidis,
\textquotedblleft The Diversity Potential of Relay Selection with Practical Channel
Estimation,\textquotedblright\ \emph{IEEE Transactions on Wireless Communications}, vol. 12, no. 2, pp. 481--493,
February 2013.

\item \textbf{N. D. Chatzidiamantis}, D. S. Michalopoulos, E. E. Kriezis, G. K. Karagiannidis, and R. Schober,
\textquotedblleft Relay Selection Protocols for Relay-Assisted Free-Space Optical
Systems,\textquotedblright\ \emph{IEEE/OSA Journal of Optical Communications and Networking}, vol. 5, no. 1, pp.
92--103, January 2013.

\item M. Matthaiou, \textbf{N. D. Chatzidiamantis}, G. K. Karagiannidis, and J. A. Nossek,
\textquotedblleft ZF Detectors over Correlated $K$ Fading MIMO Channels,\textquotedblright\
\emph{IEEE Transactions on Communications}, vol. 59, no. 6, pp. 1591--1603, June 2011.

\item \textbf{N. D. Chatzidiamantis}, A. S. Lioumpas, G. K. Karagiannidis, and S. Arnon,
\textquotedblleft Adaptive Subcarrier PSK Intensity Modulation in Free Space Optical
Systems,\textquotedblright\ \emph{IEEE Transactions on Communications}, vol. 59, no. 5, pp. 1368--1377, May 2011.

\item \textbf{N. D. Chatzidiamantis} and G. K. Karagiannidis, \textquotedblleft On the Distribution of the
Sum of Gamma-Gamma Variates and Applications in RF and Optical Wireless Communications,\textquotedblright\
\emph{IEEE Transactions on Communications}, vol. 59, no. 5, pp. 1298--1308, May 2011.

\item \textbf{N. D. Chatzidiamantis}, H. G. Sandalidis, G. K. Karagiannidis, and M. Matthaiou,
\textquotedblleft Inverse Gaussian Modeling of Turbulence-induced Fading in Free-Space Optical
Systems,\textquotedblright\ \emph{IEEE/OSA Journal of Lightwave Technology}, vol. 29, no. 10, pp. 1590--1596, May
2011.
\item M. Matthaiou, \textbf{N. D. Chatzidiamantis}, and G. K. Karagiannidis, \textquotedblleft A New Lower
Bound on the Ergodic Capacity of Distributed MIMO Systems,\textquotedblright\ \emph{IEEE Signal Processing
Letters}, vol. 18, no. 4, pp. 227--230, April 2011.

\item \textbf{N. D. Chatzidiamantis}, G. K. Karagiannidis, and M. Uysal, \textquotedblleft Generalized
Maximum-Likelihood Sequence Detection for Photon-counting Free-Space Optical Systems,\textquotedblright\
\emph{IEEE Transactions on Communications}, vol. 58, no. 12, pp. 3381--3385, December 2010.

\item M. Matthaiou, \textbf{N. D. Chatzidiamantis}, G. K. Karagiannidis, and J. A. Nossek,
\textquotedblleft On the Capacity of Generalized-$K$ Fading MIMO channels,\textquotedblright\
\emph{IEEE Transactions on Signal Processing}, vol. 58, no. 11, pp. 5939--5944, November 2010.
\item \textbf{N. D. Chatzidiamantis}, M. Uysal, T. A. Tsiftsis, and G. K. Karagiannidis,
\textquotedblleft Iterative Near Maximum-Likelihood Sequence Detection for MIMO Optical Wireless
Systems,\textquotedblright\ \emph{IEEE/OSA Journal of Lightwave Technology}, vol. 28, no. 7, pp. 1064--1070, April
2010.
\end{enumerate}

\subsection{\greektext{Διεθνή Επιστημονικά Συνέδρια}}
\latintext
\renewcommand{\labelenumi}{[C\arabic{enumi}]}
\begin{enumerate}

\item A. -A. A. Boulogeorgos, \textbf{N. D. Chatzidiamantis}, G. K. Karagiannidis, and L. Georgiadis, \textquotedblleft Energy Detection under RF impairments for Cognitive Radio,\textquotedblright\ in \emph{Proc. IEEE International Conference on Communication Workshop (IEEE ICCW)}, London, United Kingdom, 2015.
\item	\textbf{N. D. Chatzidiamantis}, L. Georgiadis, H. Sandalidis and G. K. Karagiannidis, \textquoteleft An Efficient Power Constrained Transmission Scheme for Hybrid OW/RF Systems,\textquoteright in \emph{Proc. IEEE International Conference on Communications (IEEE ICC)}, Sydney, Australia, 2014.
\item E. Matskani, \textbf{N. D. Chatzidiamantis}, L. Georgiadis, I. Koutsopoulos, L. Tassiulas, \textquoteleft The Mutual Benefits of Primary-Secondary User Cooperation in Wireless Cognitive Networks,\textquoteright in \emph{Proc. 12th International Symposium on Modeling and Optimization in Mobile, Ad Hoc, and Wireless Networks (WiOpt)}, Hammamet, Tunisia, 2014

\item K. N. Pappi, \textbf{N. D. Chatzidiamantis}, and G. K. Karagiannidis, \textquotedblleft A
Combinatorial Geometrical Approach to the Error Performance of Multidimensional Finite Lattice
Constellations,\textquotedblright\ in \emph{Proc. IEEE Wireless Communications and Networking Conference (IEEE
WCNC)}, Paris, France, 2012.

\item  \textbf{N. D. Chatzidiamantis}, G. K. Karagiannidis, E. E. Kriezis, and M. Matthaiou,
\textquotedblleft Diversity Combining in Hybrid RF/FSO Systems with PSK
Modulation,\textquotedblright\ in \emph{Proc. IEEE International Conference on Communications (IEEE ICC)}, Kyoto,
Japan, 2011.

\item  \textbf{N. D. Chatzidiamantis}, H. G. Sandalidis, G. K. Karagiannidis, and S. A. Kotsopoulos,
\textquotedblleft On the Inverse-Gaussian Shadowing,\textquotedblright\ in \emph{Proc. IEEE
International Conference on Communications (IEEE ICC)}, Kyoto, Japan, 2011.

\item D. S. Michalopoulos,  \textbf{N. D. Chatzidiamantis}, R. Schober, and G. K. Karagiannidis,
\textquotedblleft Relay Selection with Outdated Channel Estimates in Nakagami-m
Fading,\textquotedblright\ in \emph{Proc. IEEE International Conference on Communications (IEEE ICC)}, Kyoto,
Japan, 2011.

\item  \textbf{N. D. Chatzidiamantis}, D. S. Michalopoulos, E. E. Kriezis, G. K. Karagiannidis, and R. Schober,
\textquotedblleft Relay Selection in Relay-Assisted Free Space Optical Systems,\textquotedblright\ in
\emph{Proc. IEEE Global Communications Conference (IEEE GLOBECOM)}, Houston, USA, 2011.

\item M. Matthaiou,  \textbf{N. D. Chatzidiamantis}, G. K. Karagiannidis, \textquotedblleft On the Sum Rate
of ZF Detectors over Correlated $K$ Fading MIMO Channels,\textquotedblright\ in \emph{Proc. IEEE International
Conference on Acoustics, Speech and Signal Processing (ICASSP)}, Prague, Czech Republic, 2011.

\item M. Matthaiou,  \textbf{N. D. Chatzidiamantis}, H. A. Suraweera, and G. K. Karagiannidis,
\textquotedblleft Performance Analysis of Space-Time Block Codes over Generalized-$K$ Fading MIMO
Channels,\textquotedblright\ in \emph{Proc. IEEE Swedish Communication Technologies Workshop (Swe-CTW)},
Stockholm, Sweden, 2011.

\item D. S. Michalopoulos,  \textbf{N. D. Chatzidiamantis}, R. Schober, and G. K. Karagiannidis,
\textquotedblleft Diversity Loss Due to Suboptimal Relay Selection,\textquotedblright\ in \emph{Proc.
IEEE Global Communications Conference (IEEE GLOBECOM)}, Houston, USA, 2011.

\item  \textbf{N. D. Chatzidiamantis}, A. S. Lioumpas, G. K. Karagiannidis, and S. Arnon,
\textquotedblleft Optical Wireless Communications with Adaptive Subcarrier PSK Intensity
Modulation,\textquotedblright\ in \emph{Proc. IEEE Global Communications Conference (IEEE GLOBECOM)}, Miami, USA,
2010.

\item  \textbf{N. D. Chatzidiamantis}, H. G. Sandalidis, G. K. Karagiannidis, and M. Matthaiou,
\textquotedblleft A Simple Statistical Model for Turbulence-Induced Fading in Free-Space Optical
Systems,\textquotedblright\ in \emph{Proc. IEEE International Conference on Communications (IEEE ICC)}, Cape Town,
South Africa, 2010.

\item  \textbf{N. D. Chatzidiamantis}, H. G. Sandalidis, G. K. Karagiannidis, S. Kotsopoulos, and M. Matthaiou,
\textquotedblleft New Results on Turbulence Modeling for Free-Space Optical
Systems,\textquotedblright\ in \emph{Proc. International Conference on Telecommunications (ICT)}, Doha, Qatar,
2010.

\item  \textbf{N. D. Chatzidiamantis}, G. K. Karagiannidis, and D. S. Michalopoulos, \textquotedblleft On
the Distribution of the Sum of Gamma-Gamma Variates and Application in MIMO Optical Wireless
Systems,\textquotedblright\ in \emph{Proc. IEEE Global Communications Conference (IEEE GLOBECOM)}, Hawaii, USA,
2009.

\item  \textbf{N. D. Chatzidiamantis}, M. Uysal, T. A. Tsiftsis, and G. K. Karagiannidis,
\textquotedblleft EM-Based Maximum-Likelihood Sequence Detection for MIMO Optical Wireless
Systems,\textquotedblright\ in \emph{Proc. IEEE International Conference on Communications (IEEE ICC)}, Dresden,
Germany, 2009.
\end{enumerate}

\newpage
\greektext
\section{ ΥΠΟΜΝΗΜΑ ΔΗΜΟΣΙΕΥΜΕΝΩΝ ΕΡΓΑΣΙΩΝ}
\subsection{Διδακτορική Διατριβή}
\textbf{Τίτλος}: «Ανάπτυξη και Αξιολόγηση της Επίδοσης Ασύρματων Συστημάτων Επικοινωνίας Υψηλής Χωρητικότητας», Αριστοτέλειο Πανεπιστήμιο Θεσσαλονίκης, Μάρτιος 2012.\\
\textbf{Επιβλέπων}: Καθηγητής Γ. Κ. Καραγιαννίδης.
\newline\newline
Καθώς αυξάνονται οι απαιτήσεις των σύγχρονων τηλεπικοινωνιακών εφαρμογών για μεγάλους ρυθμούς μετάδοσης, καθίσταται όλο και πιο επιτακτική η ανάγκη για την ανάπτυξη τεχνολογιών επικοινωνίας υψηλής χωρητικότητας. Δύο τέτοιες ασύρματες τεχνολογίες οι οποίες έχουν προσελκύσει ιδιαίτερο ερευνητικό ενδιαφέρον τα τελευταία χρόνια είναι οι ασύρματες οπτικές (\latintext{OW}) \greektext επικοινωνίες και οι ραδιοσυχνοτικές (RF) επικοινωνίες με κατανεμημένα συστήματα πολλαπλών εισόδων και εξόδων (MIMO). Η πρώτη τεχνολογία χρησιμοποιεί οπτικά φέροντα για την μετάδοση της πληροφορίας μέσω της ατμόσφαιρας, ενώ η δεύτερη επεκτείνει την ιδέα των παραδοσιακών (\latintext{RF MIMO}) \greektext συστημάτων χρησιμοποιώντας κατανεμημένα στον χώρο συστήματα πολλαπλών κεραιών εκπομπής ή/και λήψης. Παρά όμως τα σημαντικά πλεονεκτήματα αυτών των τεχνολογιών, στην πράξη υπάρχουν σημαντικοί παράγοντες που υποβαθμίζουν την επίδοσή τους. Έτσι, η σχεδίαση και η ανάπτυξη συστημάτων επικοινωνιών που χρησιμοποιούν αυτές τις δύο τεχνολογίες μετάδοσης, καθώς και η μελέτη και αξιολόγηση της επίδοσης τους αποτελούν το κύριο αντικείμενο της παρούσας διατριβής.

Το πρώτο μέρος της παρούσας διατριβής ασχολείται με τα (\latintext{OW}) \greektext συστήματα. Αρχικά, λαμβάνοντας υπόψη ότι τα συστήματα αυτά επηρεάζονται σε μεγάλο βαθμό από το ατμοσφαιρικό κανάλι, νέες τεχνικές ανίχνευσης προτείνονται για τον περιορισμό της επίδρασης του καναλιού. Οι προτεινόμενες τεχνικές ανίχνευσης εκμεταλλεύονται την αργή χρονική μεταβλητότητα του ατμοσφαιρικού καναλιού και πραγματοποιούν ανίχνευση χωρίς να έχουν καμία πληροφορία, είτε για την στιγμιαία είτε για την στατιστική κατάσταση του καναλιού. Στην συνέχεια, αναλυτικά εργαλεία παρουσιάζονται για την αξιολόγηση της επίδοσης MIMO (\latintext{OW}) \greektext συστημάτων, βάσει μιας καινοτόμας προσέγγισης της στατιστικής του αθροίσματος \latintext{Gamma-Gamma} \greektext τυχαίων μεταβλητών. Έτσι, παράγονται νέες αναλυτικές εκφράσεις για διάφορες μετρικές επίδοσης των συστημάτων αυτών, οι οποίες μπορούν να χρησιμοποιηθούν ως αξιόπιστες εναλλακτικές στις χρονοβόρες (\latintext{Monte Carlo}) \greektext προσομοιώσεις. Ως επιπλέον μέτρο για τον περιορισμό της επίδρασης του ατμοσφαιρικού καναλιού, ακολουθεί η παρουσίαση μίας νέας τεχνικής προσαρμοσμένης μετάδοσης που μπορεί να εφαρμοστεί στα (\latintext{OW}) \greektext συστήματα. Το προτεινόμενο σχήμα μετάδοσης προσαρμόζει την τάξη της διαμόρφωσης φάσης φέροντος σύμφωνα με την στιγμιαία κατάσταση του ατμοσφαιρικού καναλιού και προσφέρει σημαντικά κέρδη στην αποδοτικότητα φάσματος. Τέλος, διερευνούνται πρωτόκολλα αναμετάδοσης για (\latintext{OW}) \greektext συστήματα που υποβοηθιούνται από αναμεταδότες και προτείνονται πρωτόκολλα επιλογής ενός αναμεταδότη. Τα προτεινόμενα σχήματα αναμετάδοσης όχι μόνο προσφέρουν καλύτερη επίδοση, αλλά αντιμετωπίζουν και τα θέματα συγχρονισμού που προκύπτουν στην περίπτωση των ανόμοιων (\latintext{OW}) \greektext ζεύξεων.

Το δεύτερο κομμάτι της παρούσας διατριβής αφορά τα (\latintext{RF}) \greektext κατανεμημένα MIMO συστήματα. Σε αυτά τα συστήματα, οι πολλαπλές κεραίες εκπομπής ή και λήψης βρίσκονται κατανεμημένες στον χώρο και εξαιτίας αυτού, τα φαινόμενα σκίασης λαμβάνονται υπόψη κατά την αξιολόγηση της επίδοσής τους. Νέα αναλυτικά όρια εξάγονται για την εργοδική χωρητικότητα των συστημάτων αυτών, θεωρώντας τα Γενικευμένο K και K μοντέλα καναλιού. Τα παραγόμενα αποτελέσματα, σε συνδυασμό με την ασυμπτωτική ανάλυση στις περιοχές των χαμηλών και υψηλών σηματοθορυβικών λόγων, αποκαλύπτουν την επίδραση των παραμέτρων συστήματος στην συνολική χωρητικότητα που επιτυγχάνεται από αυτά τα συστήματα. Επιπρόσθετα, η \latintext{Zero-Forcing (ZF)} \greektext μέθοδος ανίχνευσης διερευνάται στο πλαίσιο των κατανεμημένων MIMO συστημάτων και ένας στατιστικός χαρακτηρισμός των \latintext{Zero-Forcing (ZF)} \greektext ανιχνευτών σε MIMO κανάλια που παραμορφώνονται από γρήγορες διαλείψεις αλλά και φαινόμενα σκίασης παρουσιάζεται. Από τις αναλυτικές σχέσεις που εξάγονται, χρήσιμα συμπεράσματα προκύπτουν σχετικά με τους παράγοντες που επηρεάζουν την επίδοση των \latintext{ZF} \greektext ανιχνευτών.

\subsection{\greektext Διεθνή Επιστημονικά Περιοδικά με Κριτές}
\renewcommand{\labelenumi}{[J\arabic{enumi}]}
\begin{enumerate}\latintext
\item H. G. Sandalidis, \textbf{N. D. Chatzidiamantis}, G. D. Ntouni, and G. K. Karagiannidis, "Performance of an FSO Link using a New Mixture Composite Irradiance Model", \emph{IET Electronics Letters} To appear

\greektext{
Στην εργασία αυτή εξετάζεται η απόδοση μιας τυπικής ασύρματης οπτικής ζεύξης εξωτερικού χώρου χρησιμοποιώντας ένα σύνθετο μοντέλο για τη στατιστική περιγραφή των ατμοσφαιρικών αναταράξεων και των απωλειών ευθυγράμμισης. Το μοντέλο παρέχει μεγάλο βαθμό απλότητας και ακρίβειας και βασίζεται στην χρήση της μικτής \latintext Gamma \greektext κατανομής για την περιγραφή των αναταράξεων. Στο άρθρο παρουσιάζονται νέες αναλυτικές εκφράσεις για την πιθανότητα αποκοπής και του μέσου ρυθμού σφαλμάτων για τυπικά δυαδικά σχήματα διαμόρφωσης.}
\latintext
\item H. G. Sandalidis, \textbf{N. D. Chatzidiamantis},
and G. K. Karagiannidis, \textquotedblleft A Tractable Model for
Turbulence and Misalignment Induced Fading in Optical Wireless
Systems,\textquotedblright\ \emph{IEEE Communications Letters}, vol.
20, no. 9, pp. 1904--1907, 2016.

\greektext{
Τα πιθανοτικά μοντέλα που περιγράφουν το συνδυαστικό φαινόμενο των ατμοσφαιρικών αναταράξεων και των απωλειών ευθυγράμμισης στις επίγειες ασύρματες οπτικές ζεύξεις περιέχουν συνήθως ειδικές συναρτήσεις υψηλής τάξης γεγονός που δυσχεραίνει τη μελέτη απόδοσής τους. Με στόχο την απλοποίηση, προτείνεται στην εργασία αυτή η χρήση της μικτής κατανομής \latintext gamma \greektext ως μια ακριβής προσέγγιση της κατανομής των αναταράξεων. Στη συνέχεια εξάγεται μια καινούρια σύνθετη μικτή κατανομή που εμπεριέχει και το φαινόμενο των απωλειών ευθυγράμμισης. Η απλότητα και η ακρίβεια του μοντέλου καταδεικνύεται ακόμη και για μικρό αριθμό όρων άθροισης. Οι βασικοί στατιστικοί δείκτες εξάγονται επίσης σε κλειστή μορφή.}
\latintext
\item	A. -A. A. Boulogeorgos, \textbf{N. D. Chatzidiamantis}, and
    G. K. Karagiannidis, \textquotedblleft Energy Detection Spectrum
    Sensing Under RF imperfections,\textquotedblright\ \emph{IEEE
    Transactions on Communications}, vol. 64, no. 7, pp. 2754--2766,
    2016.

    \greektext{Οι ομόδυνοι ραδιοδέκτες με απευθείας μετατροπή} \latintext{(direct-conversion)} \greektext{προσφέρουν μία χαμηλού κόστους λύση για την ανίχνευση φάσματος σε γνωστικά ραδιοσυστήματα. Ωστόσο, αυτού του είδους οι δέκτες είναι ευάλωτοι σε ραδιοσυχνοτικές  ατέλειες, όπως ανισσοροπία στην} \latintext{I} \greektext{και στην} \latintext{Q} \greektext{συνιστώσα, μη γραμμικότητες εξαιτίας του ενισχυτή, και θόρυβο φάσης που περιορίζουν τις δυνατότητες για σωστή ανίχνευση φάσματος. Σε αυτήν την εργασία εξετάζεται η επιρροή όλων αυτών των ατελειών στην ανίχνευση φάσματος που στηρίζεται σε ανιχνευτή ενέργειας για γνωστικά συστήματα που λειτουργούν σε περιβάλλοντα με πολλαπλά κανάλια. Συγκεκριμένα, παρέχονται κλειστής μορφής εκφράσεις για τον υπολογισμό των πιθανοτήτων ανίχνευσης και ψευδούς συναγιερμού, υποθέτοντας} \latintext{Rayleigh} \greektext{διαλείψεις. Επιπλέον, επεκτείνεται η ανάλυση σε γνωστικά δίκτυα που εφαρμόζουν συνεργατική ανίχνευση, όπου οι δευτερεύοντες χρήστες υποφέρουν από διαφορετικά επίπεδα ραδιοσυχνοτικών ατελειών. Αριθμητικά και προσωμοιωτικά αποτελέσματα αποδεικνύουν την ακρίβεια της ανάλυσης και αναδεικνύουν την σημαντική επιρροή των ραδιοσυχνοτικών ατελειών στην ανίχνευση φάσματος.}
\latintext
\item A. -A. A. Boulogeorgos, \textbf{N. D. Chatzidiamantis}, and G.
    K. Karagiannidis, \textquotedblleft Spectrum Sensing with
    Multiple Primary Users over Fading Channels,\textquotedblright\
    \emph{IEEE Communications Letters}, vol. 20, no. 7, pp.
    1457--1460, 2016.

    \greektext{Εξετάζεται το αποτέλεσμα της ύπαρξης πολλαπλών πρωτεύοντων χρηστών και διαλείψεων στην ανίχνευση ελεύθερου φάσματος, χρησιμοποιώντας ανιχνευτή ενέργειας. Συγκεκριμένα, παρουσιάζονται κλειστής μορφής εκφράσεις για τις πιθανότητες ανίχνευσης και ψευδούς συναγιερμού σε ένα περιβάλλον με πολλαπλούς χρήστες υποθέτοντας} \latintext{Nakagami-m} \greektext{περιβάλλον διαλείψεων. Τα αποτελέσματα αναδεικνύουν πόσο σημαντικό είναι να λαμβάνεται υπόψη το συνολικό ασύρματο περιβάλλον, όταν εξετάζεται η ανίχνευση φάσματος με ανιχνευτή ενέργειας και επιλέγεται το κατώφλι ενέργειας.}
    \latintext
\item	P. Puri, \textbf{N. D. Chatzidiamantis}, P. Garg, M. Aggarwal,
    and G. K. Karagiannidis, \textquotedblleft Two-Way Relay
    Selection in Multiple Relayed FSO Networks,\textquotedblright\
    \emph{IEEE Wireless Communications Letters}, vol. 4, no. 5, pp.
    485--488, 2015.


    \greektext{Σε αυτή την εργασία εξετάζεται ένα ασύρματο οπτικό δίκτυο με παράλληλους αναμεταδότες. Αυτό το δίκτυο αποτελείται από πολλαπλούς} \latintext{half-duplex} \greektext{αναμεταδότες οι οποίοι ενισχύουν και προωθούν τα ενδιάμεσα οπτικά σήματα στην μεταφορά της πληροφορίας μεταξύ δύο απομακρυσμένων κόμβων. Εξετάζεται ένα πρωτόκολο μετάδοσης το οποίο επιλέγει έναν αναμεταδότη, ο οποίος μεγιστοποιεί τον μέγιστο εφικτό ρυθμό μετάδοσης του δικτύου. Υποθέτοντας ότι το οπτικό κανάλι επηρεάζεται από απώλειες διαδρομής, απώλειες λόγω έλλειψης ευθυγράμμισης και ατμοσφαιρικές, παράγονται ακριβείς και ασυμπτοτικές εκφράσεις κλειστής μορφής για τον μέγιστο εφικτό ρυθμό μετάδοσης καθώς επίσης και για τον τρόπο επιλογής αναμεταδότη.}
    \latintext
\item	\textbf{N. D. Chatzidiamantis}, L. Georgiadis,
    H. Sandalidis and G. K. Karagiannidis,
    \textquotedblleft Throughput-Optimal Link-Layer Design in Power
    Constrained Hybrid OW/RF Systems,\textquotedblright\ \emph{IEEE
    Journal on Selected Areas in Communications}, vol. 33, no. 9,
    pp. 1972--1984, 2015.

\greektext{Στην εργασία αυτή προτείνεται ένας αλγόριθμος μετάδοσης για υβριδικά συστήματα ασύρματης οπτικής-}\latintext RF \greektext{ μετάδοσης που μεγιστοποιεί τη ρυθμαπόδοση} \latintext (throughput) \greektext{λαμβάνοντας υπόψη  περιορισμούς στη συνολική ισχύ και στην ισχύ ανά ζεύξη του πομπού. Συγκεκριμένα, χρησιμοποιείται ουρά με δομή χρονοθυρίδας για την αποθήκευση των πακέτων προς αποστολή και το κανάλι μοντελοποιείται ως κανάλι διαγραφής με μεταβαλλόμενες παραμέτρους για κάθε χρονοθυρίδα σύμφωνα με αλυσίδα} \latintext Markov. \greektext{Η μεγιστοποίηση της ρυθμαπόδοσης βασίζεται στην κατάσταση της αλυσίδας και τις στατιστικές παραμέτρους του σύνθετου καναλιού σε κάθε χρονοθυριδα και γίνεται με τη μέθοδο βελτιστοποίησης} \latintext Lyapunov.  \greektext{Η απόδοση του αλγορίθμου εξετάζεται για τις περιπτώσεις ολικής και μερικής ανάδρασης στον πομπό. Η εργασία συνοδεύεται από κατάλληλα διαγράμματα τα οποία καταδεικνύουν την ευστοχία της μεθόδου στην επίτευξη ικανοποιητικής απόδοσης.}
\latintext
\item	\textbf{N. D. Chatzidiamantis}, E. Matskani, L. Georgiadis, I. Koutsopoulos, L. Tassiulas, \textquotedblleft Optimal Primary-Secondary User Cooperation Policies in Cognitive Radio Networks,\textquotedblright \emph{IEEE Transactions on Wireless Communications}, vol. 14, No 6, pp. 3443-3455, Feb. 2015.

\greektext{Στα γνωστικά ραδιοδίκτυα, οι δευτερεύοντες χρήστες έχουν την δυνατότητα να συνεργαστούν με τον πρωτεύοντα χρήστη έτσι ώστε η πιθανότητα επιτυχίας για τις μεταδόσεις του πρωτεύοντα χρήστη να αυξάνει, ενώ παράλληλα οι δευτερεύοντες χρήστες να αποκτούν πιο πολλές ευκαιρίες για μετάδοση. Ωστόσο οι δευτερεύοντες χρήστες έχουν περιορισμένα αποθέματα ισχύος, και εξαιτίας αυτού χρειάζεται να παίρνουν έξυπνες αποφάσεις να συνεργαστούν ή όχι και σε ποιο επίπεδο ισχύος προκειμένου να μεγιστοποιήσουν την ρυθμαπόδοσή τους} (\latintext{throughput}). \greektext{Οι συνεργατικές τεχνικές μετάδοσης που έχουν προταθεί έως τώρα για αυτού του είδους τα συστήματα απαιτούν την λύση ενός περιορισμένου} \latintext{Markov} \greektext{προβλήματος με άπειρο αριθμό καταστάσεων. Σε αυτήν την εργασία εξετάζεται η κλάση των πολιτικών μετάδοσης που λαμβάνουν τυχαίες αποφάσεις για την ενεργοποίηση κάποιου δευτερεύοντα χρήστη και την ισχύ μετάδοσής του σε κάθε χρονοθυρίδα βασιζόμενες μόνο στο αποτέλεσμα της ανίχνευσης φάσματος. Υποθέτοντας για τους δευτερεύοντες χρηστες ουρές με άπειρο μήκος, οι προτεινόμενες πολιτικές μετάδοσης αποδεικνύονται να πετυχαίνουν την μέγιστη ρυθμαπόδοση για τους δευτερεύοντες χρήστες, ενώ ταυτόγχρονα μεγαλώνουν την περιοχή σταθερότητας της ουράς του πρωτεύοντα χρήστη. Η δομή των βέλτιστων τεχνικών μετάδοσης παραμένει ίδια ακόμη και όταν η υπόθεση του άπειρου μήκους των ουρών των δευτερεύοντων χρηστών χαλαρώνει. Επιπλέον η ανάλυση επεκτείνεται και στις περιπτώσεις όπου η ανίχνευση φάσματος είναι ατελής. Τέλος, προτείνεται ένας χαμηλής πολυπλοκότητας κατανεμημένος αλγόριθμος μετάδοσης, ο οποίος μπορεί να εφαρμοστεί σε ρεαλιστικά σενάρια.}
\latintext
\item K. N. Pappi, \textbf{N. D. Chatzidiamantis}, and G. K. Karagiannidis, \textquotedblleft Error
Performance of Multidimensional Lattice Constellations-Part II: Evaluation over Fading
Channels,\textquotedblright\ \emph{IEEE Transactions on Communications}, vol. 61, no. 93, pp. 1099--1110, 2013.

\greektext{Στην εργασία αυτή ερευνάται η επίδοση πολυδιάστατων αστερισμών δικτυώματος με διαφορισμό χώρου σημάτων. Στο πρώτο
μέρος, μετά από μια νέα προσέγγιση συνδυαστική γεωμετρίας που είναι βασισμένη στη γεωμετρία παραλληλοτόπων,
παρουσιάστηκε μία ακριβής αναλυτική έκφραση και δύο όρια κλειστής μορφής για την πιθανότητα σφάλματος συμβόλου} \latintext{(Symbol
Error Probability - SEP)} \greektext{σε περιβάλλον προσθετικού λευκού} \latintext{Gaussian} \greektext{θορύβου} \latintext{(Additive White Gaussian Noise -
AWGN)}. \greektext{Στο παρόν μέρος ΙΙ, παρουσιάζεται μια καινούρια  αναλυτική έκφραση για την πιθανότητα λάθους πλαισίων}\latintext{
(Frame Error Probability - FEP)} \greektext{των πολυδιάστατων αστερισμών δικτυώματος σε} Nakagami-$m$ \greektext{κανάλια διαλείψεων.
Κατά αντιστοιχία με το Σφαιρικό Κάτω Όριο} \latintext{(Sphere Lower Bound - SLB)} \greektext{το οποίο αποτελεί κάτω όριο της επίδοσης} FEP
\greektext{των άπειρων αστερισμών δικτυωτού πλέγματος, προτείνουμε το Σφαιρικό Άνω Όριο} \latintext{(Sphere Upper Bound - SUB)} \greektext{για
κανάλια μπλοκ διαλείψεων. Επιπλέον,  παρουσιάζονται δύο νέα όρια για το} \latintext{FEP} \greektext{των πολυδιάστατων αστερισμών σε
κανάλια μπλοκ διαλείψεων, με ονομασίες} \latintext{MSLB} \greektext{και} \latintext{MSUB (Multiple Sphere Lower Bound, Multiple Sphere Upper Bound)}.
\greektext{Οι εκφράσεις των} \latintext{SLB} \greektext{και} \latintext{SUB} \greektext{δίνονται σε κλειστή μορφή, ενώ οι αντίστοιχες για} \latintext{MSLB} \greektext{και} \latintext{MSLB}
\greektext{δίνονται σε κλειστή μορφή για μοναδιαίο μήκος πλαισίου.}
\latintext
\item K. N. Pappi, \textbf{N. D. Chatzidiamantis}, and G. K. Karagiannidis, \textquotedblleft Error
Performance of Multidimensional Lattice Constellations-Part I: A Parallelotope Geometry Based Approach for the AWGN
Channel,\textquotedblright\ \emph{IEEE Transactions on Communications}, vol. 6, no. 3, pp. 1088--1098, 2013.

\greektext{Οι πολυδιάστατοι αστερισμοί δικτυώματος που παρουσιάζουν  διαφορισμό χώρου σημάτων} \latintext{(Signal Space Diversity - SSD)}
\greektext{έχει μελετηθεί εκτενώς για τη μετάδοση σε συστήματα μίας κεραίας και σε κανάλια εξασθένισης, εστιάζοντας στο
βέλτιστο σχεδιασμό τους για την επίτευξη του υψηλού κέρδους διαφορισμού. Σε αυτήν την διμερή σειρά εργασιών
παρουσιάζουμε μια καινούρια προσέγγιση συνδυαστικής γεωμετρίας, βασισμένη στη γεωμετρία παραλληλοτόπων, για την
αξιολόγηση της επίδοσης των πολυδιάστατων πεπερασμένων αστερισμών δικτυώματος με αυθαίρετη δομή, διάσταση και τάξη. Στο
μέρος Ι, παρουσιάζουμε μια αναλυτική έκφραση για την ακριβή πιθανότητα σφάλματος συμβόλου} \latintext{(Symbol Error Probability -
SEP)} \greektext{των πολυδιάστατων αστερισμών, και δύο νέα όρια κλειστής μορφής, τα} \latintext{MSLB} \greektext{και} \latintext{MSUB (Multiple Sphere Lower
Bound, Multiple Sphere Upper Bound)}. \greektext{Το μέρος ΙΙ επεκτείνει την ανάλυση στη μετάδοση σε κανάλια διαλείψεων, όπου οι
πολυδιάστατοι αστερισμοί δικτυώματος χρησιμοποιούνται για να καταπολεμήσουν την υποβάθμιση του σήματος λόγω των
διαλείψεων.}

\latintext
\item D. S. Michalopoulos, \textbf{N. D. Chatzidiamantis}, R. Schober, and G. K. Karagiannidis,
\textquotedblleft The Diversity Potential of Relay Selection with Practical Channel
Estimation,\textquotedblright\ \emph{IEEE Transactions on Wireless Communications}, vol. 12, no. 2, pp. 481--493,
February 2013.

\greektext{Σε αυτή την εργασία ερευνάται η τάξη διαφορισμού ενός δικτύου που εφαρμόζει αποκωδικοποίηση και επανεκπομπή με
επιλογή αναμεταδότη} (Decode-and-Forward Relay Selection) \greektext{σε κανάλια διαλείψεων} \latintext{Nakagami-$m$}, \greektext{σε περιπτώσεις
όπου εφαρμόζονται πρακτικές τεχνικές εκτίμησης καναλιών. Παρουσιάζεται ένα ενοποιημένο μοντέλο για ατελείς εκτιμήσεις
καναλιών} \latintext{(imperfect channel estimates)}, \greektext{όπου εξετάζονται από κοινού τα αποτελέσματα του θορύβου, η
χρονομεταβλητότητα και η καθυστέρηση ανατροφοδότησης. Η συσχέτιση μεταξύ των πραγματικών και των εκτιμώμενων τιμών του
καναλιού,} $r$, \greektext{εκφράζεται ως συνάρτηση του συντελεστή} \latintext{SNR}. \greektext{Οδηγώντας σε εκφράσεις κλειστής μορφής για τη
συνολική πιθανότητα διακοπής λειτουργίας σαν συνάρτηση του $r$. Η προκύπτουσα τάξη διαφορισμού και το κέρδος ισχύος
παρουσιάζουν υψηλή εξάρτηση της επίδοσης της τεχνικής της επιλογής αναμεταδότη από τη συμπεριφορά του $r$ για υψηλό}
\latintext{SNR}, \greektext{δείχνοντας έτσι την επίδραση της εκτίμησης καναλιών στη συνολική απόδοση. Δείχνεται ότι όταν οι εκτιμήσεις
καναλιών δεν ενημερώνεται συχνά στις εφαρμογές που εμπλέκουν χρονομεταβλητά κανάλια, ή όταν το ποσό της ισχύς που
διατίθεται για την εκτίμηση καναλιών δεν είναι αρκετά υψηλό, το κέρδος διαφορισμού της τεχνικής της επιλογής
αναμεταδότη μειώνεται αισθητά. Η εργασία τέλος εξετάζει πόσο γρήγορα πρέπει να τείνει η συσχέτιση $r$ στη μονάδα, καθώς
το} \latintext{SNR} \greektext{τείνει στο άπειρο, ώστε η επιλογή αναμεταδότη να μην αντιμετωπίσει απώλεια κέρδους διαφορισμού.}

\latintext
\item \textbf{N. D. Chatzidiamantis}, D. S. Michalopoulos, E. E. Kriezis, G. K. Karagiannidis, and R. Schober,
\textquotedblleft Relay Selection Protocols for Relay-Assisted Free-Space Optical
Systems,\textquotedblright\ \emph{IEEE/OSA Journal of Optical Communications and Networking}, vol. 5, no. 1, pp.
92--103, January 2013.

\greektext{Σε αυτή την εργασία, ερευνώνται τα πρωτόκολλα μετάδοσης για οπτικά συστήματα ελεύθερου χώρου υποβοηθούμενα από
αναμεταδότη} \latintext{(relay assisted free-space optical (FSO) systems)}, \greektext{στην περίπτωση που χρησιμοποιούνται πολλαπλοί
παράλληλοι αναμεταδότες και δεν υπάρχει απ' ευθείας μετάδοση μεταξύ της πηγής και του προορισμού. Σαν εναλλακτική στη
λειτουργία όλων των αναμεταδοτών, προτείνεται η επιλογή ενός μόνο} \latintext{relay} \greektext{για την επικοινωνία μεταξύ της πηγής και
του προορισμού σε κάθε βήμα μετάδοσης. Η επιλογή βασίζεται στην εκτίμηση καναλιού, που αποκτάται είτε από όλες τις
συνδέσεις είτε από τη σύνδεση} \latintext{FSO} \greektext{που χρησιμοποιήθηκε τελευταία. Με αυτό τον τρόπο ανάγκη για το συγχρονισμό των
αναμεταδοτών αποφεύγεται, ενώ γίνεται εκμετάλλευση της αργής μεταβολής του ατμοσφαιρικού καναλιού. Για τη μέθοδο
επιλογής αναμεταδότη αλλά και για τις μεθόδους αναμετάδοσης με ταυτόχρονη χρήση όλων των αναμεταδοτών, δίνονται
εκφράσεις κλειστής μορφής για την απόδοση διακοπής λειτουργίας, υποθέτοντας το Γάμμα-Γάμμα} \greektext{μοντέλο καναλιού.
Επιπλέον, με βάση τα παραγόμενα αναλυτικά αποτελέσματα, εξετάζεται το πρόβλημα βελτιστοποίησης  των οπτικών πόρων
ισχύος των} \latintext{FSO} \greektext{συνδέσεων. Προτείνονται βέλτιστες και υποβέλτιστες αλλά υπολογιστικά ελκυστικότερες λύσεις, που
οδηγούν σε αποδοτικότερα συστήματα από άποψη ισχύος.}
\latintext
\item M. Matthaiou, \textbf{N. D. Chatzidiamantis}, G. K.
    Karagiannidis, and J. A. Nossek,
\textquotedblleft ZF Detectors over Correlated $K$ Fading MIMO
Channels,\textquotedblright\
\emph{IEEE Transactions on Communications}, vol. 59, no. 6, pp.
1591--1603, June 2011.

\greektext{Αυτή η εργασία προσφέρει μία συστηματική ανάλυση για τους} \latintext{Zero-Forcing (ZF)} \greektext{δέκτες σε κανάλια πολλαπλών εισόδων πολλαπλών εξόδων που υποφέρουν και από μεγάλης-κλίμακας αλλά και από μικρής-κλίμακας διαλείψεις. Συγκεκριμένα, προτείνεται η $K$ κατανομή για να μοντελοποιήσει τις διαλείψεις τέτοιων καναλιών, υποθέτοντας επιπλέον ημι-συσχετισμένες μικρής κλίμακας διαλείψεις. Στην συνέχεια, παρουσιάζονται αναλυτικές εκφράσεις κλειστής μορφής για τον εφικτό ρυθμό μετάδοσης μαζί με ασυμπτοτικές εκφράσεις σε περιοχές υψηλού και χαμηλού σηματοθορυβικού λόγου. Στο δεύτερο μέρος αυτής της εργασίας παράγονται αναλυτικές εκφράσεις για τις πιθανότητες σφάλματος συμβόλου και διακοπής, και επιπλέον εξετάζεται η επίδοση των} \latintext{ZF} \greektext{ανιχνευτών ως προς την τάξη διαφορισμού και το κέρδος κωδικοποίησης. Ο τρόπος με τον οποίο επηρεάζεται η επίδοση των} \latintext{ZF} \greektext{δεκτών από τις παραμέτρους του μοντέλου καναλιού εξετάζεται μέσω} \latintext{Monte-Carlo} \greektext{προσομοιώσεων, οι οποίες αναδεικνύουν και την ακρίβεια της θεωρητικής ανάλυσης.}

\latintext
\item \textbf{N. D. Chatzidiamantis}, A. S. Lioumpas, G. K. Karagiannidis, and S. Arnon,
\textquotedblleft Adaptive Subcarrier PSK Intensity Modulation in Free Space Optical
Systems,\textquotedblright\ \emph{IEEE Transactions on Communications}, vol. 59, no. 5, pp. 1368--1377, May 2011.

\greektext{Προτείνεται μία προσαρμοστική τεχνική μετάδοσης για οπτικά συστήματα ελεύθερου χώρου} \latintext{(Free Space Optical - FSO)},
\greektext{που λειτουργούν σε ατμοσφαιρικές αναταράξεις και χρησιμοποιούν διαμόρφωση έντασης} \latintext{S-PSK (subcarrier Phase Shift
Keying)}. \greektext{Χρησιμοποιώντας το χαρακτηριστικό της σταθερής περιβάλλουσας του} \latintext{S-PSK}, \greektext{η προτεινόμενη τεχνική
προσφέρει αποδοτική εκμετάλλευση της χωρητικότητας των καναλιών} \latintext{FSO} \greektext{προσαρμόζοντας την τάξη διαμόρφωσης} \latintext{S-PSK},
\greektext{σύμφωνα με τη στιγμιαία κατάσταση του καναλιού και τις προκαθορισμένες απαιτήσεις όσον αφορά το ρυθμό σφάλματος}
\latintext{bit (Bit Error Rate - BER)}. \greektext{Παρουσιάζονται νέες εκφράσεις για τη φασματική αποδοτικότητα και το μέσο} \latintext{BER} \greektext{του
προτεινόμενου συστήματος} \latintext{FSO} \greektext{και εξετάζεται η απόδοση για διάφορες περιπτώσεις αναταραχών, μοντέλων ατμοσφαιρικών
αναταραχών και προαπαιτούμενου} \latintext{BER}. \greektext{H προτεινόμενη τεχνική μεταβλητού ρυθμού μετάδοσης εφαρμόζεται στα} \latintext{FSO}
\greektext{συστήματα πολλαπλών εισόδων πολλαπλών εξόδων} \latintext{(MIMO)}, \greektext{παρέχοντας επιπλέον βελτίωση στη φασματική αποδοτικότητα
καθώς ο αριθμός των παράλληλων καναλιών εκπομπής και λήψης αυξάνεται.}
\latintext
\item \textbf{N. D. Chatzidiamantis} and G. K. Karagiannidis, \textquotedblleft On the Distribution of the
Sum of Gamma-Gamma Variates and Applications in RF and Optical Wireless Communications,\textquotedblright\
\emph{IEEE Transactions on Communications}, vol. 59, no. 5, pp. 1298--1308, May 2011.

\greektext{Η κατανομή γάμμα-γάμμα} ($\Gamma\Gamma$) \greektext{έχει προσελκύσει πρόσφατα το ενδιαφέρον της ερευνητικής κοινότητας
λόγω της χρήσης της για την ανάλυση διαφόρων συστημάτων επικοινωνιών. Στα πλαίσια των ασύρματων επικοινωνιών} \latintext{RF}, \greektext{η
κατανομή} $\Gamma\Gamma$ \greektext{μπορεί να μοντελοποιήσει με ακρίβεια τη στατιστική ισχύος στα σύνθετα κανάλια με
σκίαση/διαλείψεις όπως και στα κανάλια διαδοχικών διαλείψεων πολλαπλής όδευσης. Παράλληλα στις οπτικές ασύρματες
επικοινωνίες} \latintext{(Opical Wireless - OW)}, \greektext{περιγράφει τις διακυμάνσεις της ακτινοβολίας των οπτικών σημάτων που
εξασθενούν από τις ατμοσφαιρικές συνθήκες. Αν και το} $\Gamma\Gamma$ \greektext{μοντέλο καναλιού μπορεί να περιγράψει
ικανοποιητικά την περίπτωση ασύρματων συστημάτων μίας εισόδου μιας εξόδου} \latintext{(SISO)}, \greektext{δε συμβαίνει το ίδιο στα
συστήματα πολλαπλών εισόδων πολλαπλών εξόδων} (ΜΙΜΟ), \greektext{όπου απαιτείται η γνώση της κατανομής του αθροίσματος των
ανεξαρτήτων μεταβλητών} $\Gamma\Gamma$. \greektext{Σε αυτή την εργασία, παρουσιάζεται μία νέα προσέγγιση κλειστής μορφής για
την κατανομή του αθροίσματος των ανεξάρτητων, αλλά όχι απαραιτήτως πανομοιότυπα κατανεμημένων} $\Gamma\Gamma$
\greektext{μεταβλητών. Αποδεικνύεται ότι η συνάρτηση  πυκνότητας πιθανότητας} \latintext{(PDF)} \greektext{του αθροίσματος} $\Gamma\Gamma$
\greektext{μεταβλητών μπορεί να προσεγγιστεί αποτελεσματικά είτε από την} \latintext{PDF} \greektext{μιας μόνο} $\Gamma\Gamma$ \greektext{μεταβλητής,
είτε από ένα πεπερασμένο, σταθμισμένο άθροισμα από} \latintext{PDF} \greektext{κατανομές} $\Gamma\Gamma$ \greektext{μεταβλητών. Για να
αποκαλυφθεί η σημασία της προτεινόμενης προσέγγισης, ελέγχεται η απόδοση των ασύρματων συστημάτων} \latintext{RF} \greektext{στην
περίπτωση σύνθετων διαλείψεων, καθώς επίσης και η περίπτωση} MIMO \greektext{συστημάτων που λειτουργούν σε περιβάλλον
ατμοσφαιρικής εξασθένισης.}
\latintext
\item \textbf{N. D. Chatzidiamantis}, H. G. Sandalidis, G. K. Karagiannidis, and M. Matthaiou,
\textquotedblleft Inverse Gaussian Modeling of Turbulence-induced Fading in Free-Space Optical
Systems,\textquotedblright\ \emph{IEEE/OSA Journal of Lightwave Technology}, vol. 29, no. 10, pp. 1590--1596, May
2011.

\greektext{Προτείνουμε την αντίστροφη γκαουσσιανή κατανομή, ως τη λιγότερο σύνθετη εναλλακτική λύση στο κλασσικό} log-normal
\greektext{μοντέλο, για να περιγράψουμε το φαινόμενο διαλείψεων στα οπτικά συστήματα ελευθέρου χώρου} \latintext{(FSO)} \greektext{που
λειτουργούν σε περιβάλλον ατμοσφαιρικών αναταραχών και/ή με την παρουσία φαινομένων υπολογισμού μέσου όρου} \latintext{(aperture
averaging effect)}. \greektext{Με τη βοήθεια των κατάλληλων δοκιμών, καθορίζουμε τη σειρά των τιμών του δείκτη ακτινοβολίας για
διάφορα συστήματα} \latintext{FSO} \greektext{πολλαπλών εισόδων πολλαπλών εξόδων} (MIMO), \greektext{όπου οι δύο κατανομές προσεγγίζουν η μία την
άλλη σε ένα συγκεκριμένο επίπεδο εμπιστοσύνης. Επιπλέον, εξετάζεται ο ρυθμός σφάλματος} \latintext{bit (BER)} \greektext{δύο τυπικών} MIMO
\latintext{FSO} \greektext{συστημάτων στο νέο μοντέλο αναταραχής: ένα σύστημα} \latintext{MIMO FSO} \greektext{με διαμόρφωση έντασης/άμεσης ανίχνευσης που
χρησιμοποιεί διαμόρφωση θέσης παλμού $Q$-τάξης και υιοθετεί επαναληπτική κωδικοποίηση στον πομπό και διαφορισμό ίσου
κέρδους στο δέκτη, και ένα ετερόδυνο} \latintext{MIMO FSO} \greektext{σύστημα με διαφορική διαμόρφωση ολίσθησης φάσης στον πομπό και
διαφορισμό μέγιστου κέρδους στο δέκτη.}
\latintext
\item M. Matthaiou, \textbf{N. D. Chatzidiamantis}, and G. K. Karagiannidis, \textquotedblleft A New Lower
Bound on the Ergodic Capacity of Distributed MIMO Systems,\textquotedblright\ \emph{IEEE Signal Processing
Letters}, vol. 18, no. 4, pp. 227--230, April 2011.

\greektext{Παρουσιάζεται ένα καινοτόμο και αναλυτικό κάτω όριο της εεργοδικής χωρητικότητας των διανεμημένων συστημάτων πολλών
εισόδων πολλών εξόδων} \latintext{(distributed MIMO)} \greektext{που λειτουργούν σε περιβάλον διαλείψεων} \latintext{Rayleigh}/\greektext{Λογαριθμοκανονικών
θεωρώντας χωρική συσχέτιση διπλής πλευράς. Το προτεινόμενο κάτω όριο ισχλύει για πεπερασμένο αριθμό κεραιών και για
κάθε ποσοστό σήμα προς θόρυβο} \latintext{(SNR)}. \greektext{Επιπλέον, εκτελείται μια λεπτομερής χαμηλού}-\latintext{SNR} \greektext{ανάλυση που παρέχει
χρήσιμες πληροφορίες για τις παραμέτρους των συστημάτων σε} MIMO \greektext{χωρητικότητα.}

\latintext
\item \textbf{N. D. Chatzidiamantis}, G. K. Karagiannidis, and M. Uysal, \textquotedblleft Generalized
Maximum-Likelihood Sequence Detection for Photon-counting Free-Space Optical Systems,\textquotedblright\
\emph{IEEE Transactions on Communications}, vol. 58, no. 12, pp. 3381--3385, December 2010.

\greektext{Σε αυτή την εργασία, ερευνώνται οι μέθοδοι για} \latintext{on-off} \greektext{κωδικοποίηση} \latintext{(on-off keying OOK)} \greektext{μέτρησης φωτονίων
σε Οπτικά Συστήματα Ελεύθερου Χώρου} \latintext{(FSO)} \greektext{σε περιβάλλον διαλείψεων εξαιτίας ατμοσφαιρικών αναταραχών, θεωρώντας
ότι δεν υπάρχει πληροφορία της κατάστασης του καναλιού στο δέκτη. Για να επαναποκτήσουν  την απώλεια επίδοσης που
συνδέεται με τη σύμβολο-προς-σύμβολο ανίχνευση σε μια τέτοια περίπτωση, θεωρούνται τεχνικές ανίχνευσης ακολουθίας , που
αξιοποιούν τη χρονική συσχέτιση των} \latintext{FSO} \greektext{καναλιών. Στις περισσότερες πρακτικές εφαρμογές η βέλτιστη ανίχνευση
ακολουθίας μέγιστης πιθανοφάνειας} \latintext{(MLSD)} \greektext{είναι ασύμφορη της υψηλής πολυπλοκότητας υπολογισμού. Προτείνεται ένας
υποβέλτιστος κανόνας απόφασης χαμηλής πολυπλοκότητας που βασίζεται στη γενικευμένη εκτίμηση ακολουθίας μέγιστης
πιθανοφάνειας. Η προτεινόμενη μέθοδος επιτρέπει τη ανίχνευση ακολουθιών μεγάλου μήκους που ήταν απαγορευτικές για
χρήση} \latintext{MLSD}, \greektext{όταν δε χρησιμοποιείται καμία γνώση του καναλιού.}
\latintext
\item M. Matthaiou, \textbf{N. D. Chatzidiamantis}, G. K. Karagiannidis, and J. A. Nossek,
\textquotedblleft On the Capacity of Generalized-$K$ Fading MIMO channels,\textquotedblright\
\emph{IEEE Transactions on Signal Processing}, vol. 58, no. 11, pp. 5939--5944, November 2010.

\greektext{Αυτή η εργασία ερευνά την εργοδική χωρητικότητα των συστημάτων πολλαπλών εισόδων πολλαπλών εξόδων} (MIMO) \greektext{σε
περιβάλλον γενικευμένων-$K$} \latintext{(generalized-K)} \greektext{διαλείψεων. Χρησιμοποιώντας θεωρία μεγιστοποίησης εξάγεται ένα
αναλυτικό όριο χωρητικότητας που μπορεί να χρησιμοποιηθεί για οποιεσδήποτε τιμές του λόγου σήματος προς θόρυβο και για
οποιοδήποτε αριθμό κεραιών. Επιπλέον, εξάγονται απλές προσεγγίσεις του ορίου για υψηλές τιμές} \latintext{SNR} \greektext{και δείχνεται
ότι οι επιπτώσεις των διαλείψεων μικρής και μεγάλης κλίμακας είναι αποζευγμένες. Παρόμοια ανάλυση πραγματοποιείται για}
MIMO \greektext{συστήματα σε περιβάλλον $Κ$-διαλείψεων, που αποτελούν υποπερίπτωση των γενικευμένων $Κ$-διαλείψεων και μπορούν
να εξετασθούν με χρήση θεωρίας πινάκων} \latintext{Wishart}.

\latintext
\item \textbf{N. D. Chatzidiamantis}, M. Uysal, T. A. Tsiftsis, and G. K. Karagiannidis,
\textquotedblleft Iterative Near Maximum-Likelihood Sequence Detection for MIMO Optical Wireless
Systems,\textquotedblright\ \emph{IEEE/OSA Journal of Lightwave Technology}, vol. 28, no. 7, pp. 1064--1070, April
2010.

\greektext{Ένας σημαντικός παράγοντας περιορισμού της επίδοσης στα επίγεια οπτικά ασύρματα συστήματα} \latintext{(OW)} \greektext{είναι οι
διαλείψεις λόγω ατμοσφαιρικών συνθηκών. Αξιοποιώντας τους πρόσθετους βαθμούς ελευθερίας στη χωρική διάσταση, ο
συνδυασμός πολλαπλών πομπών} \latintext{laser} \greektext{με πολλούς δέκτες παρέχει μία αποτελεσματική λύση καταπολεμώντας τις διαλείψεις.
Αν και τα συστήματα πολλαπλών εισόδων πολλαπλών εξόδων} (MIMO) \greektext{έχουν μελετηθεί εκτενώς τα προηγούμενα χρόνια, το
μεγαλύτερο κομμάτι της έρευνας περιορίζεται στην σύμβολο προς σύμβολο αποκωδικοποίηση. Η} \latintext{MLSD} \greektext{τεχνική
εκμεταλλεύεται τη χρονική συσχέτιση της αναταραχής που προκαλεί τις διαλείψεις και υπόσχεται περαιτέρω κέρδη επίδοσης.
Σε αυτή την εργασία, μελετάται η τεχνική} \latintext{MLSD} \greektext{για} \latintext{MIMO OW} \greektext{συστήματα διαμόρφωσης έντασης/άμεσης-ανίχνευσης σε}
\latintext{log-normal} \greektext{κανάλια ατμοσφαιρικής εξασθένισης. Ακόμα και με μία τεχνική χαμηλής τάξης διαμόρφωσης, όπως η} \latintext{OOK},
\greektext{που χρησιμοποιείται συνήθως στα} \latintext{OW} \greektext{συστήματα, η πολυπλοκότητα της} \latintext{MLSD} \greektext{μπορεί να είναι απαγορευτική. Γι'
αυτό παρουσιάζουμε έναν επαναληπτικό ανιχνευτή ακολουθίας που βασίζεται στον αλγόριθμο μεγιστοποίησης προσδοκίας} \latintext{
(expectation-maximization)}. \greektext{H πολυπλοκότητα του προτεινόμενου αλγορίθμου θεωρείται μικρότερη από τη κατ' ευθείαν
εκτίμηση της} \latintext{log-likehood} \greektext{συνάρτησης και είναι ανεξάρτητη από τη στατιστική εξασθένισης του καναλιού.}

\end{enumerate}

\subsection{\greektext Διεθνή Επιστημονικά Συνέδρια}
\renewcommand{\labelenumi}{[C\arabic{enumi}]}
\center
\begin{enumerate}\latintext
\item A. -A. A. Boulogeorgos, \textbf{N. D. Chatzidiamantis}, G. K.
    Karagiannidis, and L. Georgiadis, \textquotedblleft Energy
    Detection under RF impairments for Cognitive
    Radio,\textquotedblright\ in \emph{Proc. IEEE International
    Conference on Communication Workshop (IEEE ICCW)}, London,
    United Kingdom, 2015.

    \greektext{Οι ομόδυνοι ραδιοδέκτες με απευθείας μετατροπή \en{(direct-conversion)} προσφέρουν μία χαμηλού κόστους λύση για την ανίχνευση φάσματος σε γνωστικά ραδιοσυστήματα. Ωστόσο, αυτού του είδους οι δέκτες είναι ευάλωτοι σε ραδιοσυχνοτικές  ατέλειες, όπως ανισσοροπία στην \en{I} και στην \en{Q} συνιστώσα, μη γραμμικότητες εξαιτίας του ενισχυτή, και θόρυβο φάσης που περιορίζουν τις δυνατότητες για σωστή ανίχνευση φάσματος. Σε αυτήν την εργασία εξετάζεται η επιρροή όλων αυτών των ατελειών στην ανίχνευση φάσματος που στηρίζεται σε ανιχνευτή ενέργειας για γνωστικά συστήματα που λειτουργούν σε περιβάλλοντα με πολλαπλά κανάλια. Συγκεκριμένα, παρέχονται κλειστής μορφής εκφράσεις για τον υπολογισμό των πιθανοτήτων ανίχνευσης και ψευδούς συναγιερμού, υποθέτοντας \en{Rayleigh} διαλείψεις. Αριθμητικά και προσωμοιωτικά αποτελέσματα αποδεικνύουν την ακρίβεια της ανάλυσης και αναδεικνύουν την σημαντική επιρροή των ραδιοσυχνοτικών ατελειών στην ανίχνευση φάσματος.}
    \latintext
\item	\textbf{N. D. Chatzidiamantis}, L. Georgiadis,
    H. Sandalidis and G. K. Karagiannidis, \textquoteleft
    An Efficient Power Constrained Transmission Scheme for Hybrid
    OW/RF Systems,\textquoteright in \emph{Proc. IEEE International
    Conference on Communications (IEEE ICC)}, Sydney, Australia,
    2014.

    \greektext{Η εργασία αυτή διερευνά τεχνικές μετάδοσης στο επίπεδο ζεύξης για υβριδικά συστήματα ασύρματης οπτικής μετάδοσης (\en{OW}) και μετάδοσης στην περιοχή ραδιοσυχνοτήτων με περιορισμούς στη συνολική ισχύ όσο και στην ισχύ ανά ζεύξη του πομπού. Συγκεκριμένα, διατυπώνεται ένα στοχαστικό πρόβλημα βελτιστοποίησης που λαμβάνει υπόψη τον αριθμό πακέτων που φθάνει τυχαία και αποθηκεύεται σε ένα κατάλληλο μοντέλο ουράς και τα επίπεδα ισχύος της μετάδοσης που απαιτείται σε κάθε ζεύξη. Η μελέτη έχει σαν στόχο το σχεδιασμό κατάλληλης πολιτικής ελέγχου που μεγιστοποιεί τη ρυθμαπόδοση (\en{throughput}) χρησιμοποιώντας μεθόδους βελτιστοποίησης \en{Lyapunov}. Ο προτεινόμενος αλγόριθμος ικανοποιεί τον επιδιωκόμενο στόχο απόδοσης αξιοποιώντας ικανοποιητικά και τις δυο ζεύξεις με βάση τους περιορισμούς στην κατανάλωση ισχύος.}
    \latintext
\item E. Matskani, \textbf{N. D. Chatzidiamantis}, L.
    Georgiadis, I. Koutsopoulos, L. Tassiulas,
    \textquoteleft The Mutual Benefits of Primary-Secondary User
    Cooperation in Wireless Cognitive Networks,\textquoteright in
    \emph{Proc. 12th International Symposium on Modeling and
    Optimization in Mobile, Ad Hoc, and Wireless Networks (WiOpt)},
    Hammamet, Tunisia, 2014

    \greektext{Στα γνωστικά ραδιοδίκτυα, οι δευτερεύοντες χρήστες έχουν την δυνατότητα να συνεργαστούν με τον πρωτεύοντα χρήστη έτσι ώστε η πιθανότητα επιτυχίας για τις μεταδόσεις του πρωτεύοντα χρήστη να αυξάνει, ενώ παράλληλα οι δευτερεύοντες χρήστες να αποκτούν πιο πολλές ευκαιρίες για μετάδοση. Ωστόσο οι δευτερεύοντες χρήστες έχουν περιορισμένα αποθέματα ισχύος, και εξαιτίας αυτού χρειάζεται να παίρνουν έξυπνες αποφάσεις να συνεργαστούν ή όχι και σε ποιο επίπεδο ισχύος προκειμένου να μεγιστοποιήσουν την ρυθμαπόδοσή τους (\en{throughput}). Οι συνεργατικές τεχνικές μετάδοσης που έχουν προταθεί έως τώρα για αυτού του είδους τα συστήματα απαιτούν την λύση ενός περιορισμένου \en{Markov} προβλήματος με άπειρο αριθμό καταστάσεων. Σε αυτήν την εργασία εξετάζεται η κλάση των πολιτικών μετάδοσης που λαμβάνουν τυχαίες αποφάσεις για την ενεργοποίηση κάποιου δευτερεύοντα χρήστη και την ισχύ μετάδοσής του σε κάθε χρονοθυρίδα βασιζόμενες μόνο στο αποτέλεσμα της ανίχνευσης φάσματος. Οι προτεινόμενες πολιτικές αποδεικνύονται να πετυχαίνουν τους ίδιους ρυθμούς μετάδοσης για τους δευτερεύοντες χρήστες που πετυχαίνουν πιο γενικές πολιτικές μετάδοσης, ενώ ταυτόγχρονα μεγαλώνουν την περιοχή σταθερότητας της ουράς του πρωτεύοντα χρήστη. Τέλος, προτείνεται ένα κατανεμημένο πρωτόκολλο μετάδοσης που στηρίζεται στον κατανεμημένο υπολογισμό των πιθανοτήτων της προτεινόμενης κλάσης των πολιτικών μετάδοσης και εφαρμόζεται σε πρακτικά σενάρια μετάδοσης.}
\latintext
    \item K. N. Pappi,  \textbf{N. D. Chatzidiamantis}, and G. K. Karagiannidis, \textquotedblleft A
Combinatorial Geometrical Approach to the Error Performance of Multidimensional Finite Lattice
Constellations,\textquotedblright\ in \emph{Proc. IEEE Wireless Communications and Networking Conference (IEEE
WCNC)}, Paris, France, 2012.

\greektext{Στην παρούσα εργασία παρουσιάζεται μία νέα μέθοδος αξιολόγησης της επίδοσης των πολυδιάστατων πεπερασμένων
αστερισμών δικτυώματος σε κανάλια λευκού προσθετικού} \latintext{Gaussian} \greektext{θορύβου} \latintext{(AWGN)}, \greektext{η οποία χρησιμοποιεί
συνδυαστική γεωμετρία. Πιο συγκεκριμένα, παρουσιάζεται μία ακριβής αναλυτική έκφραση για την πιθανότητα σφάλματος
συμβόλου} (SEP) \greektext{των πολυδιάστατων αστερισμών, η οποία στη συνέχεια χρησιμοποιείται για την εξαγωγή ενός ακριβούς
κάτω ορίου για το} SEP, \greektext{το οποίο ονομάζεται} \latintext{Multiple Sphere Lower Bound (MSLB)}. \greektext{Το} \latintext{MSLB} \greektext{μπορεί να
χρησιμοποιηθεί για αστερισμούς με τυχαία δομή, διάσταση και μέγεθος, ενώ μπορεί να επεκταθεί και στην περίπτωση
καναλιών διαλείψεων, όπου οι πολυδιάστατοι αστερισμοί χρησιμοποιούνται για να καταπολεμήσουν τις επιπτώσεις των
διαλείψεων.}
\latintext
\item  \textbf{N. D. Chatzidiamantis}, G. K. Karagiannidis, E. E. Kriezis, and M. Matthaiou,
\textquotedblleft Diversity Combining in Hybrid RF/FSO Systems with PSK
Modulation,\textquotedblright\ in \emph{Proc. IEEE International Conference on Communications (IEEE ICC)}, Kyoto,
Japan, 2011.

\greektext{Στην παρούσα εργασία παρουσιάζεται μία νέα αρχιτεκτονική υβριδικών ασύρματων συστημάτων ραδιοσυχνοτήτων} \latintext{(Radio
Frequency - RF)} / \greektext{οπτικής επικοινωνίας ελεύθερου χώρου} \latintext{(free space optical - FSO)}, \greektext{χωρίς ανατροφοδότηση και
χωρίς πληροφορία της κατάστασης του καναλιού στον πομπό. Με δεδομένη την παραδοχή ότι τα} \latintext{RF} \greektext{και} \latintext{FSO} \greektext{συστήματα
που λειτουργούν στα 60} \latintext{GHz} \greektext{υποστηρίζουν τους ίδιους ρυθμούς μετάδοσης, η προτεινόμενη εφαρμογή μεταδίδει τα
δεδομένα και από τα δύο κανάλια, χρησιμοποιώντας κοινή διαμόρφωση ολίσθησης φάσης} \latintext{(Phase Shift Keying - PSK)} \greektext{και
συνδυάζει τα σήματα των δύο καναλιών στο δέκτη σύμβολο{-}προς{-}σύμβολο. Εξετάζονται δύο συχνά χρησιμοποιούμενες
τεχνικές συνδυαστικού διαφορισμού, ο συνδυασμός με επιλογή} \latintext{(Selection Combining - SC)} \greektext{και ο συνδυασμός μέγιστου
λόγου} \latintext{(Maximal Ratio Combining - MRC)}, \greektext{ενώ εξάγονται χρηστικές αναλυτικές προσεγγίσεις για το ρυθμό σφάλματος} \latintext{bit
(BER)}. \greektext{Η εξέταση του συστήματος σε διάφορες καιρικές συνθήκες και αποστάσεις των κόμβων αποκαλύπτει ότι η
προτεινόμενη αρχιτεκτονική αξιοποιεί πλήρως τη συμπληρωματική φύση των επικοινωνιών} \latintext{RF} \greektext{και} \latintext{FSO}, \greektext{ακόμη και
όταν το ένα από τα δύο κανάλια δεν είναι διαθέσιμο. Επιπλέον, η σύγκριση των τεχνικών αναδεικνύει την τεχνική} \latintext{MRC}
\greektext{ως τη βέλτιστη, προσφέροντας μεγαλύτερα κέρδη στις αποστάσεις των κόμβων σε σχέση με την} \latintext{SC} \greektext{τεχνική.}
\latintext
\item  \textbf{N. D. Chatzidiamantis}, H. G. Sandalidis, G. K. Karagiannidis, and S. A. Kotsopoulos,
\textquotedblleft On the Inverse-Gaussian Shadowing,\textquotedblright\ in \emph{Proc. IEEE
International Conference on Communications (IEEE ICC)}, Kyoto, Japan, 2011.

\greektext{Η αντίστροφη} \latintext{Gaussian} \greektext{κατανομή} \latintext{(Inverse Gaussian - IG)} \greektext{προτάθηκε πρόσφατα ως μία εναλλακτική στην πιο
περίπλοκη} \latintext{log-normal} \greektext{κατανομή, για την περιγραφή των φαινομένων σκίασης. Η εργασία αυτή εξετάζει την επίδοση
ασύρματων συστημάτων σε} \latintext{IG} \greektext{κανάλια διαλείψεων. Δίνονται εκφράσεις κλειστής μορφής για την πιθανότητα διακοπής και
το μέσο ρυθμό σφάλματος} \latintext{bit} \greektext{για διάφορες διαμορφώσεις. Επιπλέον αναλύεται η επίδοση των δεκτών συνδυασμού μέγιστου
    λόγου} \latintext{(maximal ratio combining)} \greektext{και συνδυασμού με επιλογή} \latintext{(selection combining).
\latintext
\item D. S. Michalopoulos,  \textbf{N. D. Chatzidiamantis}, R. Schober, and G. K. Karagiannidis,
\textquotedblleft Relay Selection with Outdated Channel Estimates in Nakagami-m
Fading,\textquotedblright\ in \emph{Proc. IEEE International Conference on Communications (IEEE ICC)}, Kyoto,
Japan, 2011.
\latintext
\greektext{Σε αυτή την εργασία εξετάζεται η επίπτωση της εκτίμησης προηγούμενης κατάστασης του καναλιού} \latintext{(outdated channel
state information)} \greektext{στην επίδοση της αποκωδικοποίησης και επανεκπομπής με επιλογή αναμεταδότη, σε λειτουργία σε
κανάλια διαλείψεων} \latintext{Nakagami}-$m$. \greektext{Εξάγονται εκφράσεις για την πιθανότητα διακοπής, ως συνάρτηση της συσχέτισης
μεταξύ της εκτίμησης και της πραγματικής τιμής του καναλιού. Μελετώνται επίσης το κέρδος διαφορισμού και το κέρδος
κωδικοποίησης, ενώ αποκαλύπτεται μεγάλη εξάρτηση της τάξης διαμόρφωσης από την προαναφερθείσα συσχέτιση. Αναδεικνύεται
ότι όταν η εκτίμηση του καναλιού δεν είναι ακριβής και επίκαιρη, η επίδοση του συστήματος επιλογής αναμεταδότη, με βάση
την τάξη διαφορισμού, είναι ίδια με αυτή ενός συστήματος, στο οποίο μόνο ένας αναμεταδότης είναι διαθέσιμος.}
\latintext
\item  \textbf{N. D. Chatzidiamantis}, D. S. Michalopoulos, E. E. Kriezis, G. K. Karagiannidis, and R. Schober,
\textquotedblleft Relay Selection in Relay-Assisted Free Space Optical Systems,\textquotedblright\ in
\emph{Proc. IEEE Global Communications Conference (IEEE GLOBECOM)}, Houston, USA, 2011.

\greektext{Εξετάζονται τεχνικές εκπομπής σε οπτικές επικοινωνίες ελεύθερου χώρου} \latintext{(FSO)} \greektext{υποβοηθούμενες από αναμεταδότη,
στην περίπτωση που λειτουργούν πολλαπλοί παράλληλοι αναμεταδότες και δεν υπάρχει απευθείας σύνδεση πομπού και δέκτη. Ως
εναλλακτική στην περίπτωση που είναι ενεργοί όλοι οι αναμεταδότες και λειτουργούν παράλληλα, προτείνεται μία τεχνική
στην οποία επιλέγεται και συμμετέχει στην επικοινωνία πομπού - δέκτη σε κάθε χρονοθυρίδα μόνο ένας αναμεταδότης, με
κριτήριο την πληροφορία της κατάστασης όλων των διαθέσιμων καναλιών} \latintext{(Channel State Information - CSI)}. \greektext{Επομένως
δεν είναι αναγκαίος ο συγχρονισμός της λειτουργίας όλων των αναμεταδοτών, ενώ αξιοποιείται η αργή μεταβολή του
ατμοσφαιρικού καναλιού. Χρησιμοποιώντας το μοντέλο διαλείψεων} \latintext{Gamma-Gamma}, \greektext{εξάγονται νέες αναλυτικές εκφράσεις
κλειστού τύπου για την πιθανότητα διακοπής, τόσο για την περίπτωση της επιλογής ενός μόνο αναμεταδότη, όσο και για την
περίπτωση της λειτουργίας πολλών παράλληλων αναμεταδοτών.}
\latintext
\item M. Matthaiou,  \textbf{N. D. Chatzidiamantis}, G. K. Karagiannidis, \textquotedblleft On the Sum Rate
of ZF Detectors over Correlated $K$ Fading MIMO Channels,\textquotedblright\ in \emph{Proc. IEEE International
Conference on Acoustics, Speech and Signal Processing (ICASSP)}, Prague, Czech Republic, 2011.

\greektext{Στην εργασία αυτή παρουσιάζεται μία αναλυτική μελέτη του αθροιστικού ρυθμού ανιχνευτών εξαναγκασμού στο μηδέν}\latintext{
(Zero-Forcing (ZF) detectors)} \greektext{σε σύνθετα κανάλια πολλαπλών εισόδων - πολλαπλών εξόδων} (MIMO). \greektext{Για τη
μοντελοποίηση των σύνθετων διαλείψεων θεωρείται η γενική} $Κ$ \greektext{κατανομή} (\latintext{Rayleigh/Gamma} \greektext{κατανομή), ενώ
θεωρείται επιπλέον η γενική περίπτωση των ημι-συσχετισμένων διαλείψεων μικρής κλίμακας. Εξάγονται νέες ακριβείς
αναλυτικές εκφράσεις για τον αθροιστικό ρυθμό μετάδοσης που μπορεί να επιτευχθεί, ενώ δίνονται και ασυμπτωτικές
εκφράσεις για την περίπτωση χαμηλών τιμών του λόγου σήματος προς θόρυβο} \latintext{(SNR)}. \greektext{Παράλληλα, εξάγονται νέες εκφράσεις
κλειστής μορφής για τα άνω και κάτω όρια του αθροιστικού ρυθμού μετάδοσης, οι οποίες είναι ακριβείς για κάθε τιμή του}
SNR.
\latintext
\item M. Matthaiou,  \textbf{N. D. Chatzidiamantis}, H. A. Suraweera, and G. K. Karagiannidis,
\textquotedblleft Performance Analysis of Space-Time Block Codes over Generalized-$K$ Fading MIMO
Channels,\textquotedblright\ in \emph{Proc. IEEE Swedish Communication Technologies Workshop (Swe-CTW)},
Stockholm, Sweden, 2011.

\greektext{Στην παρούσα εργασία εξετάζεται η επίδοση ορθογώνιων μπλοκ κωδίκων χώρου-χρόνου} \latintext{(orthogonal STBC)} \greektext{για
συστήματα πολλαπλών εισόδων - πολλαπλών εξόδων} (MIMO) \greektext{που λειτουργούν σε περιβάλλον γενικευμένων-$Κ$ διαλείψεων.
Το θεωρούμενο μοντέλο διαλείψεων είναι γενικό, καθώς περιλαμβάνει τόσο τις διαλείψεις μικρής κλίμακας (που
μοντελοποιούνται με την κατανομή} \latintext{Nakagami}-$m$) \greektext{όσο και τις διαλείψεις μεγάλης κλίμακας (που μοντελοποιούνται με
την κατανομή} \latintext{Gamma}). \greektext{Στη συνέχεια εξάγονται νέες αναλυτικές εκφράσεις για την κατά} \latintext{Shannon} \greektext{χωρητικότητα, αλλά
και ασυμπτωτικές προσεγγίσεις για την περιοχή υψηλών τιμών του λόγου σήματος προς θόρυβο} \latintext{(SNR)}. \greektext{Επιπλέον
παρουσιάζονται ακριβείς εκφράσεις αλλά και προσεγγίσεις πρώτης τάξης για τη οριακή πιθανότητα διακοπής και το ρυθμό
σφάλματος συμβόλου} \latintext{(SER)}. \greektext{Στη συνέχεια ποσοτικοποιείται η επίδοση των} \latintext{STBC} \greektext{με τη μορφή της τάξης διαφορισμού
και του κέρδους κωδικοποίησης.}
\latintext
\item D. S. Michalopoulos,  \textbf{N. D. Chatzidiamantis}, R. Schober, and G. K. Karagiannidis,
\textquotedblleft Diversity Loss Due to Suboptimal Relay Selection,\textquotedblright\ in \emph{Proc.
IEEE Global Communications Conference (IEEE GLOBECOM)}, Houston, USA, 2011.

\greektext{Η επίδοση των συστημάτων επιλογής αναμεταδότη υποβαθμίζεται όταν οι εκτιμήσεις του καναλιού που χρησιμοποιούνται
για την επιλογή δεν είναι ακριβείς. Θεωρώντας ότι η εκτίμηση του καναλιού πραγματοποιείται με πρακτικές μεθόδους
εκτίμησης, ποσοτικοποιούνται οι επιπτώσεις της μη ακριβούς εκτίμησης του καναλιού} \latintext{(imperfect Channel State Information
- imperfect CSI)} \greektext{στην τάξη διαφορισμού που επιτυγχάνεται με την επιλογή αναμεταδότη. Η παρούσα ανάλυση περιλαμβάνει
σε ένα ενοποιημένο μοντέλο τόσο τις επιπτώσεις του θορύβου, των χρονομεταβλητών καναλιών όσο και της καθυστέρησης
επιστροφής στην εκτίμηση του καναλιού που χρησιμοποιείται για την επιλογή του αναμεταδότη. Με βάση αυτό το μοντέλο, η
συσχέτιση $\rho$ μεταξύ των εκτιμήσεων και των πραγματικών τιμών του καναλιού εκφράζεται ως συνάρτηση του λόγου σήματος
προς θόρυβο} \latintext(SNR)}, \greektext{επιτρέποντας την αξιολόγηση της συμπεριφοράς της πιθανότητας διακοπής σε υψηλές τιμές του} \latintext{SNR}.
\greektext{Η εξαγόμενη έκφραση για την τάξη διαφορισμού αποκαλύπτει ότι η συμπεριφορά της συσχέτισης $\rho$ σε υψηλές τιμές}
\latintext{SNR} \greektext{επηρεάζει σοβαρά την ασυμπτωτική επίδοση των συστημάτων επιλογής αναμεταδότη.}
\latintext
\item  \textbf{N. D. Chatzidiamantis}, A. S. Lioumpas, G. K. Karagiannidis, and S. Arnon,
\textquotedblleft Optical Wireless Communications with Adaptive Subcarrier PSK Intensity
Modulation,\textquotedblright\ in \emph{Proc. IEEE Global Communications Conference (IEEE GLOBECOM)}, Miami, USA,
2010.

\greektext{Προτείνεται μια προσαρμοστική τεχνική μετάδοσης για συστήματα ασύρματων επικοινωνιών} \latintext{(optical wireless - OW)},
\greektext{που λειτουργούν σε ατμοσφαιρικές αναταράξεις και χρησιμοποιούν διαμόρφωση έντασης} \latintext{S-PSK (subcarrier Phase Shift
Keying)}. \greektext{Χρησιμοποιώντας το χαρακτηριστικό της σταθερής περιβάλλουσας του} \latintext{S-PSK}, \greektext{η προτεινόμενη τεχνική
προσφέρει αποδοτική εκμετάλλευση της χωρητικότητας των} \latintext{OW} \greektext{καναλιών προσαρμόζοντας την τάξη διαμόρφωσης} \latintext{S-PSK},
\greektext{σύμφωνα με τη στιγμιαία κατάσταση του καναλιού και τις προκαθορισμένες απαιτήσεις όσον αφορά το ρυθμό σφάλματος}
\latintext{bit (Bit Error Rate - BER)}. \greektext{Παρουσιάζονται νέες εκφράσεις για τη φασματική αποδοτικότητα και το μέσο} \latintext{BER} \greektext{του
προτεινόμενου συστήματος} \latintext{OW} \greektext{και εξετάζεται η απόδοση για διάφορες περιπτώσεις αναταραχών, μοντέλων ατμοσφαιρικών
αναταραχών και προαπαιτούμενου} \latintext{BER}.
\latintext
\item  \textbf{N. D. Chatzidiamantis}, H. G. Sandalidis, G. K. Karagiannidis, and M. Matthaiou,
\textquotedblleft A Simple Statistical Model for Turbulence-Induced Fading in Free-Space Optical
Systems,\textquotedblright\ in \emph{Proc. IEEE International Conference on Communications (IEEE ICC)}, Cape Town,
South Africa, 2010.

\greektext{Προτείνουμε την αντίστροφη γκαουσσιανή κατανομή, ως τη λιγότερο σύνθετη εναλλακτική για το κλασσικό} \latintext{log-normal}
\greektext{μοντέλο, για να περιγράψουμε το φαινόμενο διαλείψεων στα οπτικά συστήματα ελευθέρου χώρου} \latintext{(FSO)} \greektext{που
λειτουργούν σε περιβάλλον ατμοσφαιρικών αναταραχών και/ή με την παρουσία φαινομένων υπολογισμού μέσου όρου ανοίγματος}\latintext{
(aperture averaging effect)}. \greektext{Με τη βοήθεια των κατάλληλων δοκιμών, καθορίζουμε την περιοχή τιμών του δείκτη
σπινθηρισμού, όπου οι δύο κατανομές προσεγγίζουν η μία την άλλη σε ένα συγκεκριμένο επίπεδο εμπιστοσύνης. Εξετάζεται η
αποδοτικότητα του καινούριου μοντέλου εξάγοντας αναλυτικές εκφράσεις για τον υπολογισμό του ρυθμού σφάλματος} \latintext{bit (BER)}
\greektext{δύο τυπικών} \latintext{FSO} \greektext{συστημάτων: ενός συστήματος} \latintext{FSO} \greektext{με διαμόρφωση έντασης/άμεσης ανίχνευσης που χρησιμοποιεί
διαμόρφωση θέσης παλμού $Μ$-τάξης, και ένα ετερόδυνο} \latintext{FSO} \greektext{σύστημα με διαφορική διαμόρφωση ολίσθησης φάσης.}
\latintext
\item N. D. Chatzidiamantis, H. G. Sandalidis, \textbf{G. K. Karagiannidis}, S. Kotsopoulos, and M. Matthaiou,
\textquotedblleft New Results on Turbulence Modeling for Free-Space Optical
Systems,\textquotedblright\ in \emph{Proc. International Conference on Telecommunications (ICT)}, Doha, Qatar,
2010.

\greektext{Σε αυτή την εργασία προτείνεται ένα στατιστικό μοντέλο καναλιού, το} \latintext{Double-Weibull}, \greektext{για να περιγράψει τις
διακυμάνσεις ακτινοβολίας στα οπτικά συστήματα ελευθέρου χώρου} \latintext{(FSO)} \greektext{σε περιβάλλον μέτριας και ισχυρής
ατμοσφαιρικής αναταραχής. Το προτεινόμενο στοχαστικό μοντέλο είναι βασισμένο στη θεωρία σπινθηρισμού και παράγεται με
τη βοήθεια δύο τυχαίων μεταβλητών που ακολουθούν την κατανομή} \latintext{Weibull}. \greektext{Δίνονται εκφράσεις κλειστής μορφής της
συνάρτησης πυκνότητας πιθανότητας και αθροιστικής συνάρτησης πιθανότητας σε όρους της συνάρτησης} \latintext{Meijer's} $G$.
\greektext{Επίσης γίνεται σύγκριση του νέου μοντέλου με το κλασικό γάμμα-γάμμα μοντέλο και εξετάζεται η ακρίβειά του με τη
βοήθεια προσομοιώσεων, τόσο για σφαιρικά όσο και για επίπεδα κύματα. Τέλος, εκτιμάται η επίδοση ενός} \latintext{FSO}
\greektext{συστήματος σε περιβάλλον αναταραχής} \latintext{Double-Weibull} \greektext{και δίνονται εκφράσεις κλειστής μορφής για το ρυθμό
σφάλματος} \latintext{bit} \greektext{υποθέτοντας} \latintext{On-Off} \greektext{διαμόρφωση έντασης με απευθείας ανίχνευση, και για την πιθανότητα διακοπής
λειτουργίας.}
\latintext
\item  \textbf{N. D. Chatzidiamantis}, G. K. Karagiannidis, and D. S. Michalopoulos, \textquotedblleft On
the Distribution of the Sum of Gamma-Gamma Variates and Application in MIMO Optical Wireless
Systems,\textquotedblright\ in \emph{Proc. IEEE Global Communications Conference (IEEE GLOBECOM)}, Hawaii, USA,
2009.

\greektext{Παρουσιάζεται μια καινούρια μεθοδολογία για την ακριβή προσέγγιση της κατανομής του αθροίσματος των ανεξάρτητων
αλλά όχι απαραίτητα πανομοιότυπα κατανεμημένων μεταβλητών γάμμα-γάμμα} \latintext{(GG)}, \greektext{μέσω εκφράσεων κλειστής μορφής.
Δείχνεται ότι η συνάρτηση πυκνότητας πιθανότητας} \latintext{(PDF)} \greektext{του αθροίσματος} \latintext{GG} \greektext{μπορεί να εκτιμηθεί ικανοποιητικά
από την} \latintext{PDF} \greektext{μιας μόνο κατανομής} \latintext{GG}, \greektext{είτε από ένα πεπερασμένο σταθμισμένο άθροισμα από} \latintext{PDF} \greektext{που ακολουθούν
την κατανομή} \latintext{GG}. \greektext{Εξακριβώνοντας την εγκυρότητα αυτού του αποτελέσματος, μελετάται η επίδοση οπτικών συστημάτων
πολλαπλών εισόδων πολλαπλών εξόδων} (MIMO), \greektext{ενώ δίνονται προσεγγιστικές εκφράσεις κλειστής μορφής για σημαντικά
μεγέθη που αφορούν την επίδοση.}
\latintext
\item  \textbf{N. D. Chatzidiamantis}, M. Uysal, T. A. Tsiftsis, and G. K. Karagiannidis,
\textquotedblleft EM-Based Maximum-Likelihood Sequence Detection for MIMO Optical Wireless
Systems,\textquotedblright\ in \emph{Proc. IEEE International Conference on Communications (IEEE ICC)}, Dresden,
Germany, 2009.

\greektext{Ένας σημαντικός παράγοντας περιορισμού της επίδοσης στα επίγεια οπτικά ασύρματα συστήματα} \latintext{(OW)} \greektext{είναι οι
διαλείψεις λόγω ατμοσφαιρικών συνθηκών. Αξιοποιώντας τους πρόσθετους βαθμούς ελευθερίας στη χωρική διάσταση, ο
συνδυασμός πολλαπλών πομπών} \latintext{laser} \greektext{με πολλούς δέκτες ανοίγματος παρέχει μία αποτελεσματική λύση καταπολεμώντας τις
διαλείψεις. Αν και τα συστήματα πολλαπλών εισόδων πολλαπλών εξόδων} (MIMO) \greektext{έχουν μελετηθεί εκτενώς τα προηγούμενα
χρόνια, το μεγαλύτερο κομμάτι της έρευνας περιορίζεται στην σύμβολο προς σύμβολο αποκωδικοποίηση. Η τεχνική ανίχνευσης
ακολουθίας μέγιστης πιθανοφάνειας} \latintext{(MLSD)} \greektext{εκμεταλλεύεται τη χρονική συσχέτιση της αναταραχής που προκαλεί τις
διαλείψεις και υπόσχεται περαιτέρω κέρδη επίδοσης. Σε αυτή την εργασία, μελετάται η τεχνική} \latintext{MLSD} \greektext{για} \latintext{MIMO OW}
\greektext{συστήματα διαμόρφωσης έντασης{/}άμεσης{-}ανίχνευσης σε} log-normal \greektext{κανάλια ατμοσφαιρικής εξασθένισης. Ακόμα και
με μία τεχνική χαμηλής τάξης διαμόρφωσης, όπως η} \latintext{On-off Keying} \greektext{ (OOK), που χρησιμοποιείται συνήθως στα} \latintext{OW}
\greektext{συστήματα, η πολυπλοκότητα της} \latintext{MLSD} \greektext{μπορεί να είναι απαγορευτική. Γι' αυτό παρουσιάζουμε έναν επαναληπτικό
ανιχνευτή ακολουθίας που βασίζεται στον αλγόριθμο μεγιστοποίησης προσδοκίας} \latintext{(expectation-maximization)}. \greektext{H
πολυπλοκότητα του προτεινόμενου αλγορίθμου θεωρείται μικρότερη από την κατευθείαν εκτίμηση της} \latintext{log-likelihood}
\greektext{συνάρτησης.}

\end{enumerate}




\end{document}
